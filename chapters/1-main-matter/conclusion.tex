%%%%%%%%%%%%%%%%%%%%%%%%%%%%%%%%%%%%%%%%%%%%%%%%%%%%%%%%%%%%%%%%%%%%%%%%%%%%%%%
\chapter{Discussion and Planning}
%%%%%%%%%%%%%%%%%%%%%%%%%%%%%%%%%%%%%%%%%%%%%%%%%%%%%%%%%%%%%%%%%%%%%%%%%%%%%%%

%%%%%%%%%%%%%%%%%%%%%%%%%%%%%%%%%%%%%%%%%%%%%%%%%%%%%%%%%%%%%%%%%%%%%%%%%%%%%%%
\section{Summary of Insights}
%%%%%%%%%%%%%%%%%%%%%%%%%%%%%%%%%%%%%%%%%%%%%%%%%%%%%%%%%%%%%%%%%%%%%%%%%%%%%%%
The primary objective of this paper is to determine a guidance policy used by a spacecraft in optimally navigating a small celestial body or protoplanet with initially unknown or highly unconstrained physical characteristics. The motive behind this goal is to subsequently minimise the time to characterise a small celestial body and maximise the efficacy of observations in characterising its physical characteristics as discussed in \autoref{chap:introduction}. A key subfield of \gls{AI}: \gls{RL} is progressively becoming more popular in autonomous space exploration as discussed in \autoref{ssec:rl-in-space}.

%%%%%%%%%%%%%%%%%%%%%%%%%%%%%%%%%%%%%%%%%%%%%%%%%%%%%%%%%%%%%%%%%%%%%%%%%%%%%%%
\subsection{Small bodies (and protoplanets) in the Solar System}\label{ssec:insight:small-bodies}
%%%%%%%%%%%%%%%%%%%%%%%%%%%%%%%%%%%%%%%%%%%%%%%%%%%%%%%%%%%%%%%%%%%%%%%%%%%%%%%

Samples from the distribution of small bodies (and protoplanets) across the Solar System were discussed in \autoref{sec:historical_missions_to_small_bodies}, selected by their presence in past and future planned space missions. It was mentioned in the \autoref{chap:introduction} that the term ``small body'' refers to all Solar System objects that avoided accretion by the Sun, a planet, or any of their moons thereof, as defined by Planetary Science and Astrobiology Decadal Survey. These include the categories: ``Martian moons, Trojan asteroids, irregular satellites, comets, Centaurs, and trans-Neptunian objects in the Kuiper belt, Scattered Disk, Detached Scattered Disk, and Oort cloud''. Furthermore, the subclassification of asteroids was encountered during \autoref{sec:historical_missions_to_small_bodies}, where the three composition classes of asteroids were seen to contain C-, M- and S-types. These classes were named referring to the primarily identified carbonaceous, metallic and silicaceous compositions and relate to how far an asteroid formed from the Sun.

Space missions are often met with unexpected opportunities to observe and characterise small bodies by the very nature of the objective: exploration. For example, during Galileo's flyby of 243 Ida in 1993 discussed in \autoref{ssec:dactyl}, it was discovered that Ida had a small moon which was later named Dactyl. Opportunities such as that of Dactyl's discovery present the importance of furthering autonomous exploration technologies to increase the scientific return of space missions, improving our understanding of the Solar System.

%%%%%%%%%%%%%%%%%%%%%%%%%%%%%%%%%%%%%%%%%%%%%%%%%%%%%%%%%%%%%%%%%%%%%%%%%%%%%%%
\subsection{Hardware limitations in spaceflight}\label{ssec:insight:hardware-limitations}
%%%%%%%%%%%%%%%%%%%%%%%%%%%%%%%%%%%%%%%%%%%%%%%%%%%%%%%%%%%%%%%%%%%%%%%%%%%%%%%
The time restraints of \gls{DNN} optimisation on-board a spacecraft is determined by the current state-of-the-art space qualified \glspl{GPU}.
As of now, there is only one commercially available \gls{GPU} that is space qualified. Aitech's S-A1760 radiation-characterised space \gls{AI} general purpose \gls{GPU} (which uses the NVIDIA Jetson TX2i system-on-module) system with 256 \gls{CUDA} cores, 1.3 \glspl{TFLOP} and 32 GB of \gls{VRAM} discussed in \autoref{ssec:dl-space}. The value of 1.3 \glspl{TFLOP} provides this work with a baseline for the computational time requirements for evaluating and optimising a \gls{DNN} on board a spacecraft. Furthermore, the value of 32GB of \gls{VRAM} provides a baseline for the memory constraints of a (or multiple) \glspl{DNN} size onboard a spacecraft. This memory constraint is directly related to the table filling strategy of backpropagation discussed at the end of \autoref{ssec:mlps}. It should be noted that the values given for the \glspl{FLOP} specifications of \glspl{GPU} are generally provided in 32-bit floating point numerical representations, as 64-bit precision is not generally used for \gls{ML} applications involving \glspl{DNN}. The reason for this is that given a consistent \gls{DNN} architecture, the evaluation time and storage requirements for all weights and biases increase storage requirements on the \gls{VRAM} of a \gls{GPU}. Therefore, one will either need to reduce their model size which results in a loss of capacity and therefore ability to capture complexity as discussed in \autoref{ssec:capacity_overfitting_underfitting} or use a \gls{GPU} with more \gls{VRAM}. The loss of precision is not a concern for this work as the impact on simulated results due to the uncertainty in the physical characteristics of a small celestial body is of a far higher magnitude. Often even half-precision (16-bit) is used for \gls{ML} applications involving \glspl{DNN}, improving the efficiency of training. On a side note, this does not mean that the precision of observations and estimated parameters will be held to the same restriction, as this is outside the loop of the \gls{DNN} optimisation. It would also be no hard task to convert a model that has learning using a lower precision to that of a higher, which would potentially tackle a majority of the optimisation time over a large dataset, improving the efficiency of the training process. The secondary training period would be aimed at reducing the quantization noise incurred from the initial training period with low precision as noted in \autoref{dl:optimisation}.

%%%%%%%%%%%%%%%%%%%%%%%%%%%%%%%%%%%%%%%%%%%%%%%%%%%%%%%%%%%%%%%%%%%%%%%%%%%%%%%
\subsection{Parametrisation of State and Environment}\label{ssec:insight:state-environment}
%%%%%%%%%%%%%%%%%%%%%%%%%%%%%%%%%%%%%%%%%%%%%%%%%%%%%%%%%%%%%%%%%%%%%%%%%%%%%%%

Two key concepts of reinforcement learning discussed in \autoref{ssec:key_concepts_and_terminology} were the states and actions of a learning agent which parametrise the environment influencing the agent, and the agent's influence the environment respectively. The problem that can arise from over-parametrisation of these relates to that of the ``Curse of dimensionality'' coined by Richard E. Bellman mentioned in \autoref{chap:dynamic_modelling} and \autoref{point:curse_of_dimensionality_1}. This phenomenon arises from analysing and organising data in high-dimensional space. When dimensionality increases, the volume of the space increases dramatically such that the available data becomes sparse. The typical effect of this in machine learning is felt when attempting to model patterns in the feature space of the experiences provided to the algorithm. As the feature space dimensionality grows, the quantity of data required for accurate generalisation grows exponentially. For this reason, it was discussed in \autoref{chap:dynamic_modelling} that the rotational dynamics of the spacecraft would not be included in the action space and that the thrust control action space of the spacecraft would be parametrised by the RSW frame (described in \autoref{fig:relative_orbital_frame}). Not only does this subsequently reduce the requirements on the simulated data required for training, but it also reduces the complexity of the learning task, allowing for a more efficient training process. This reduction in complexity arises from the omission of the RSW frame transformation from the learning task. This is best illustrated by the Gaussian \gls{VOP} \autoref{eq:vop:gauss:kep}, whereby it is immediately clear how to influence the osculating orbit in the RSW action space. \autoref{ssec:state:swarm} further discusses the parametrisation of the state space of a swarm of spacecraft. It was mentioned that there are two primary methods of tackling a varying state space dimension: create multiple models for different state sizes, or use mean embeddings of the state space which requires only one model. The latter method is favoured for this work as it is more efficient in terms of computational time and memory requirements. The mean embedding method is discussed in \autoref{ssec:state:swarm} \cite{Huttenrauch2018}.

%, rather than needing to

%%%%%%%%%%%%%%%%%%%%%%%%%%%%%%%%%%%%%%%%%%%%%%%%%%%%%%%%%%%%%%%%%%%%%%%%%%%%%%%
\subsection{The conflicting optimums of risk and reward}\label{ssec:insight:conflicting-optimums}
%%%%%%%%%%%%%%%%%%%%%%%%%%%%%%%%%%%%%%%%%%%%%%%%%%%%%%%%%%%%%%%%%%%%%%%%%%%%%%%

A perceivable obstacle in the effective autonomous characterisation of a small body is the potentially conflicting goals of risk mitigation and science return. In \autoref{ssec:expansion_in_spherical_harmonics} it was seen that the influence of spherical harmonic influence drops with the inverse of the distance from the centre of the body $r$, which indicates that closer flybys of the body would improve the observability of the contribution of the harmonic coefficients to the gravitational acceleration experienced by a spacecraft. At the same time, given highly unconstrained parameters, this would prove to be a highly risky manoeuvre, as collision probability would be higher. Therefore there is a trade-off between the risk of collision and the discernible influence of higher-order influences of the gravitational potential model. The best possible solution to this problem would be to consider a policy which is Pareto optimal between the two objectives, allowing for different types of missions between scientific and time-critical potential Earth-impactor characterisation, where multiple expendable spacecraft could be used. An approach to addressing this problem would be the use of a \gls{MORL} algorithm, which is discussed in \autoref{ssec:multi-objective}.

%%%%%%%%%%%%%%%%%%%%%%%%%%%%%%%%%%%%%%%%%%%%%%%%%%%%%%%%%%%%%%%%%%%%%%%%%%%%%%%
\subsection{Autonomy and Earth-based observations}\label{ssec:insight:autonomy-earth-based}
%%%%%%%%%%%%%%%%%%%%%%%%%%%%%%%%%%%%%%%%%%%%%%%%%%%%%%%%%%%%%%%%%%%%%%%%%%%%%%%

The use of Earth-based observations to characterise small bodies is a well-established technique and is used in conjunction with spacecraft observations to provide a more complete picture of the small body. By definition of autonomy, Earth-based observations should be controlled during the training and validation of autonomous policies used by spacecraft. The use Earth-based observations can be used to provide a priori information about the small body, which can be used to inform the autonomous characterisation of the small body, however throughout the mission, we aim at using only the observations made by the spacecraft. This is to ensure that the spacecraft can autonomously characterise the small body without the need for external intervention. The degree to which Earth-based observations are used becomes a primary controllable parameter of this work. 

% \subsection{Reinforcement Learning Algorithm}

% - With consideration of the computational feasibility of space-grade hardware, what reinforcement learning algorithm should be used to determine the optimal navigation policy?

% \subsection{Estimation Algorithm}

% - With consideration of the computational feasibility of space-grade hardware and the training time of a reinforcement learning algorithm, what estimation algorithm should be used to determine the optimal navigation policy?

% \subsection{State Representation}

% - What orbital state representation provides the learning algorithm with the fastest improvement in performance?

% \subsection{Reward \& Punishment}

% - What information metric should be used to measure optimality in a policy's performance in its ability to characterise a small body?
% - What lookahead sliding window should be used for calculating conjunction probabilities?
% - Should the agent be punished for its risk of leaving the vicinity of the small body or is the lack of its ability to characterise with increased distance from the small body sufficient influence of equivalent behaviour?

% \subsection{Information vs. Preservation}

% \subsection{Swarm Navigation}

% - The constant state dimension for each member of the Swarm, and how to weight the nearest neighbour's relative state in the Swarm's state representation (Collision Risk vs. Spacial Weighting). It may be beneficial to hierarchically cluster other members of the swarm by their relative state, or collision risk and then weight the nearest neighbour's relative state by the cluster it belongs to.

% \textbf{State}


\newpage\section{Research Definition}

\subsection{Research Problem}

The established research problem, given the literature survey and identified gaps in contemporary practices, in the autonomous navigation, guidance and control of a spacecraft in the vicinity of a small body (or protoplanet) is discussed in \autoref{chap:introduction}. The primary research problem is the lack of work towards autonomous spacecraft characterisation algorithms in the contexts of science return and time-critical characterisation of potential Earth-impactors.

% :

% \autoref{chap:introduction}

% \begin{quote}

% \end{quote}

%%%%%%%%%%%%%%%%%%%%%%%%%%%%%%%%%%%%%%%%%%%%%%%%%%%%%%%%%%%%%%%%%%%%%%%%%%%%%%%
\subsection{Research Question(s)}
%%%%%%%%%%%%%%%%%%%%%%%%%%%%%%%%%%%%%%%%%%%%%%%%%%%%%%%%%%%%%%%%%%%%%%%%%%%%%%%

Given the research problem, the research question focuses on a quantitative evaluation of the rapidly converging fields of machine learning and space exploration, with a holistic approach across the two fields. The \textbf{top-level} research question is as follows:

\begin{quote}
    What practical techniques from the progressively intersecting fields of reinforcement learning and space exploration yield an optimal guidance and control policy for the generalised characterisation of a small body (or protoplanet) with initially highly unconstrained physical characteristics by (an) autonomous spacecraft?
\end{quote}

This question, although specific in its objective, can be broken down further given the surveyed literature which has identified key components of the relevant ``practical techniques''. These are ordered by a bottom-up approach, starting with the most fundamental choice of techniques and working up to the most abstract.

\begin{quote}
    What are the appropriate probability models that define the physical characteristics of small bodies and protoplanets as observed in our Solar System, is there observable dependence between subsets of these characteristics?
\end{quote}

The importance of capturing the underlying distribution of small bodies and protoplanets in our Solar System is paramount to the success of the research question. To be able to ``generalise'' as discussed in \autoref{ssec:capacity_overfitting_underfitting} is a key desired ability in machine learning. Furthermore, it is a component of paramount importance in the research question as we aim to obtain a policy that can perform as expected on previously unseen samples of small bodies and protoplanets in future discoveries by providing an appropriate ``training'' landscape of possible encounters. Without this, any quantitative analyses of the derived policies, chosen learning algorithms, or required observations and estimation algorithms, are meaningless, as it is not a true reflection of the performance of these components in the real world. After all, there is no such thing as a ``Free Lunch'' in machine learning as discussed in \autoref{ssec:capacity_overfitting_underfitting}: ``our goal is to understand what kinds of distributions are relevant to the real world that an \gls{AI} agent experiences.'' Alternatively, we could infinitely draw from distributions which exceed the bounds of the real world, but this would result in wasted computational resources, a large portion of its hypothesis space would be wasted on capturing patterns which do not extend to reality. Finally, in regions of the distribution which \textbf{are} representative of the real world, the policy will perform poorly, having only under fitted the complexity of that region with a subset of its hypothesis space. This effect is illustrated by \autoref{fig:underfitting} as opposed to the desired effect in \autoref{fig:appropriate-capacity}, upon considering the omission of a subset of data points in the key region of the minima.

\begin{quote}
    What metric is most effective in evaluating the performance of the spacecraft guidance and control policy in characterising the physical characteristics of a given small body or protoplanet?
\end{quote}

% \begin{quote}
%     What primary performance metric is most effective in evaluating the performance of the spacecraft guidance and control policy in characterising the physical characteristics of a given small body or protoplanet?
% \end{quote}


The performance metric determines what behaviour is learned by the algorithm and was discussed in \autoref{sec:ML-performance} in the context of regression and classification machine learning tasks. The problem with these contemporary performance metrics is that their metrics do not measure differents in probability distributions, but rather differences between scalars. Furthermore, the reward signal during the rollout (or trajectory) of a reinforcement learning case, will diminish over time, as the algorithm tends towards constraining the higher-order less influential harmonics of the gravitational potential. As the higher-order harmonics are generally of lower magnitudes, this would provide a diminished reward signal at later parts of the trajectory, if a method such as the mean squared error was used. Alternatively, we can utilise a metric which is more suited to information in evaluating prior beliefs and posterior beliefs, such as the Kullback-Leibler divergence. This is discussed in \autoref{sec:information_theory}.


\begin{quote}
    Given computational restraints imposed by space-qualified hardware, what reinforcement learning algorithm, or category thereof, is best suited to an optimal guidance and control policy of a spacecraft?
\end{quote}

Discussed in detail in \autoref{ssec:insight:hardware-limitations}, the limitations of hardware in spacecraft systems are primarily due to the need for radiation-hardened components, which are generally more expensive and less powerful than their non-radiation-hardened counterparts. \gls{VRAM} space on \glspl{GPU} determine the sizes of models in \gls{DL}, and \gls{FLOP} specification provides a metric for the evaluation of \gls{DL}. These two metrics should be considered during the feasibility study of different \gls{RL} algorithms.

\begin{quote}
    Do swarm-based smaller spacecraft systems provide a more efficient means of characterising a small body or protoplanet than a single spacecraft?
\end{quote}

Given the trend of smaller spacecraft systems in the near future, it is important to consider the potential benefits of a swarm-based approach to the research question. Their benefits of reduced cost and increased observability in absence of Earth-based tracking are clear. A solution to implementing a varying swarm state space representation is discussed in \autoref{ssec:state:swarm}.





% In order to evaluate a performance metric

% - assume PPO
% - eval metric
% - eval punishments
% - eval conjunctions
% - eval lookahead
% - eval state representation


% - continual learning onboard, as characterisation of the body is carried out?




% \begin{quote}
%     What class of reinforcement learning algorithm performs be used to determine the optimal guidance and control policy?
% \end{quote}


% - With consideration of the computational feasibility of space-grade hardware, what reinforcement learning algorithm should be used to determine the optimal navigation policy?

% \begin{quote}
%     What class of reinforcement learning algorithm performs be used to determine the optimal guidance and control policy?
% \end{quote}

% \begin{enumerate}
%     \item What estimation algorithm should be used to determine the optimal navigation policy?
%     \item What orbital state representation provides the learning algorithm with the fastest improvement in performance?
%     \item What information metric should be used to measure optimality in a policy's performance in its ability to characterise a small body?
%     \item What lookahead sliding window should be used for calculating conjunction probabilities?
%     \item Should the agent be punished for its risk of leaving the vicinity of the small body or is the lack of its ability to characterise with increased distance from the small body sufficient influence of equivalent behaviour?
% \end{enumerate}

%%%%%%%%%%%%%%%%%%%%%%%%%%%%%%%%%%%%%%%%%%%%%%%%%%%%%%%%%%%%%%%%%%%%%%%%%%%%%%%
\newpage\section{Thesis Plan}
%%%%%%%%%%%%%%%%%%%%%%%%%%%%%%%%%%%%%%%%%%%%%%%%%%%%%%%%%%%%%%%%%%%%%%%%%%%%%%%

\begin{enumerate}
    \item Establish the raw dataset (L0: Raw) representative of the probability distribution for the physical characteristics and model parameters across all categories of catalogued small bodies and protoplanets in the Solar System. \textcolor{blue}{[2 weeks]}
          \begin{enumerate}
              \item \textcolor{orange}{[week 1/2]} Collect all up-to-date available data on the physical characteristics\footnote{\url{https://ssd.jpl.nasa.gov/tools/sbdb_query.html}} and shape models\footnote{\label{note1}\url{https://3d-asteroids.space/asteroids/}}\footnote{\label{note2}\url{https://3d-asteroids.space/comets/}} of all catalogued small bodies in the Solar System and document sources. 
              \item \textcolor{orange}{[week 1/2]} Clean the collected data, assimilating all physical characteristics and paths to the corresponding small body shape files (where available), and document the format of the data.
              \item \textcolor{orange}{[week 1/2]} Select a parameter set to be used across all catalogued small bodies containing the common set of parameters defining the physical characteristics, rotation, shape and gravitational potential models (i.e. set degree and order for spherical harmonics, or select quantity of mascons). Leave samples with missing data in the dataset.
          \end{enumerate}
    At this point, it will become clear (after the preliminary inspection of available data) that there is a significant amount of missing data in the dataset, specifically in the gravitational potential model, and that the data may not be uniformly distributed across all classes of small bodies.

    \item For bodies with no constrained gravitational potential models, use a polyhedral gravitational potential model \cite{Werner1996}, calculate the equivalent spherical harmonic coefficients (for a constant predetermined degree and order) for either constant \cite{Werner1997}, linearly varying \cite{Chen2019a} or the polynomial contrast \cite{Chen2019b} assumption of density distribution (L1: Dataset processed with assumed mass distribution of bodies). For bodies where mass is not constrained, draw it from a probability distribution representative of that body's known composition, spectral or orbit class (L2: Dataset processed with synthesized mass and assumed mass distribution of small bodies). This will form \textcolor{blue}{[2-3.5 weeks]}
    \begin{enumerate}
        \item \textcolor{orange}{[week 1/2]} Implement (and verify) code for the expansion of spherical harmonics coefficients from a given polyhedral gravitational potential model for constant, linear varying and polynomial contrast density distribution.
        \item \textcolor{orange}{[week 1/2]} Investigate bodies with unconstrained masses and determine what sets of probability distributions should be available to draw from in order to synthesize the mass of the bodies with unconstrained mass, representative of the population probability distribution.
        \item \textcolor{orange}{[week 2/2]} For each small body, homogenise the body shape models to the highest vertices count available, and add the results to the dataset. This can be achieved with the use of Blender with a surface subdivision modifier and is done in order to avoid capturing non-realistic frequencies in the higher-order and -degree spherical harmonics due to the polyhedral shape. 
        \item \textcolor{orange}{[week 2/2]} For each small body, calculate the equivalent spherical harmonic coefficients for a given degree and order for each density distribution assumption, adding the results to the dataset.
    \end{enumerate}
    
    There are two cases which may arise from the currently processed dataset:
    \begin{itemize}
        \item \textbf{Case A:} The higher-order spherical harmonics exhibit very low influence on the gravitational potential model for a significant part of the dataset. This would pose an imbalance in the distribution of cases encountered by the reinforcement learning algorithm, as some cases may be vastly more complex than others. A solution to this is a ``numerical approach to synthesizing realised asteroid surfaces from morphological parameters'' by Li et al. \cite{Li2022}. This step would involve either reaching out to the authors for their code base (preferable) or implementing it using the paper. This would form a third level in the data processing, involving the synthesis of fine shape augmentation (L3: Dataset processed with synthesized mass, assumed mass distribution, and synthesized fine features of small bodies) \textcolor{orange}{[+2.0 week]}
        \item \textbf{Case B:} The higher-order spherical harmonics exhibit sufficient influence on the gravitational potential model for a sufficient part of the dataset. There is no need to modify the shape models to induce more representation by higher-order spherical harmonic coefficients.
    \end{itemize}

    At this point, a dataset of a consistent set of parameters will be available for all small bodies in the Solar System with constrained body shapes (representative of the population distribution) ready to be used in the environments of a small body characterisation reinforcement learning problem.

    \item Establish the estimation framework for the spacecraft in the small body characterisation problem. The intervals at which least-squares solutions are provided from Earth-based observation become a free parameter in the reinforcement learning environment, in order to further tune the level of autonomy of the spacecraft and find more compare effectiveness of policies. The spacecraft (or swarm) will utilise sequential estimation using a selected Kalman filter. \textcolor{blue}{[4 weeks]}
    \begin{enumerate}
        \item \textcolor{orange}{[week 1/4]} Set up the observables for Earth-based tracking of the spacecraft (or swarm) for both \gls{DSN} and/or \gls{ESTRACK} stations. The integration intervals used for Doppler tracking observables will be assumed as 60 and 1000 seconds \cite{Dirkx2019}. 
        \item Establish satellite-to-satellite tracking (SST) observables \cite{Kakihara2016, Hill2006} for autonomous spacecraft (or swarm) tracking. 
        \item \textcolor{orange}{[week 1/4]} Construct realistic noise functions for the observables, based on the \gls{DSN} noise characteristics \cite[p.~33]{Thorton2003}.
        \item \textcolor{orange}{[week 2/4]} Investigate the state of affairs of the \gls{EKF} in Tudat, and if necessary, refactor (or add another) standalone interface (if need be) and validate with unit tests.
        \item \textcolor{orange}{[week 3/4]} Write functional standalone code for the \gls{UKF}, and validate using unit tests. Using the \gls{UKF} in practice will require components in Tudat to be deep-copyable, which is currently not a feature. This will require the implementation of deep copy constructors for the dynamics simulator, and possibly other components.
        \item \textcolor{orange}{[week 3/4]} Ensure that the Kalman filters implemented have copy constructors to facilitate deep copying when used in the reinforcement learning environment.
        \item \textcolor{orange}{[week 3/4]} Test the estimation framework with Marie's orbital design (or with Eros if a binary asteroid system is deemed to entail work exceeding the scope of the work) and analyse the results of the constrained parameters over time.
    \end{enumerate}
    % \item \textcolor{orange}{[week 2/4]} Document the previous steps and manage references for the final report.
    % to use the filter at a higher level as part of the later implemented open-loop simulation API.

    \item Compare performance metrics by running a global optimisation routine in the test case used in the previous step to determine the optimal spacecraft trajectory. This will be used as a baseline (closed-loop solution) for the reinforcement learning policy (open-loop solution) to compare against. The optimisation problem will use the improved Sims Flanaga formulated by Yam et al. \cite{Yam2010} \textcolor{blue}{[4 weeks]}
    \begin{enumerate}
        \item \textcolor{orange}{[week 1/4]} Evaluate the implementation of the Sundman transform in Tudat, in order to propagate in the s-domain, where $s_0=0$ and $ds=dt/r$. Attempt this by only controlling $dt$ given the dependent variable of the position of the spacecraft. This may require the implementation of a new propagator in Tudat, and the implementation of a new state derivative model for the spacecraft dynamics. If the estimated workload exceeds the scope of the work, time domain propagation will be used instead.
        \item \textcolor{orange}{[week 2/4]} Implement the calculation of collision probability for the spacecraft with the small body which will form part of the reward function \cite{Patera2001}.
        \item \textcolor{orange}{[week 3/4]} Formulate the optimisation problem by parametrising the design space to be thrust in the RSW frame with $\Delta{V}$ impulses across an assumed mission duration.
        \item \textcolor{orange}{[week 3/4]} Implement and investigate alternative performance metrics and constraints by running a given global optimisation routine (for example, particle swarm optimisation (PSO) favoured by Izzo).
        \item \textcolor{orange}{[week 4/4]} Evaluate and compare the results of performance metrics and constraints for the closed-loop solution.
    \end{enumerate}

    \item Use \gls{PPO}\footnote{\url{https://openai.com/blog/openai-baselines-ppo/}} \cite{Schulman2017} as a baseline reinforcement learning algorithm to evaluate characterisation performance metrics and their resultant optimal policies for guidance and control of an autonomous spacecraft in the vicinity of a small body or protoplanet generated by distributions obtained in (1). \textcolor{blue}{[2 weeks]}
          \begin{enumerate}
              \item \textcolor{orange}{[week 1/2]} Implement an interface for closed-loop control of the dynamics simulator of Tudat (this is already done on a private branch for the most part).
              \item \textcolor{orange}{[week 1/2]} Implement unit tests for arbitrary spacecraft guidance and control policies with the new interface and ensure consistency in the results with the validated open-loop interface of the dynamics simulator.
              \item \textcolor{orange}{[week 2/2]} Construct a problem environment for the reinforcement learning algorithm ensuring compatibility with the \gls{PPO} algorithm in the environment.
              \item \textcolor{orange}{[week 2/2]} Perform the reinforcement learning optimisation and evaluate the results of the performance metrics and constraints of the policies.
          \end{enumerate}

    \item Explore different reinforcement learning algorithms and compare their performance, computational complexity, and storage complexity. Compare the results to the current limitations of hardware in space. \textcolor{blue}{[4 weeks]}

    \item Investigate the use of multi-objective reinforcement learning for risk vs. rewards tradeoffs \cite{Moffaert2014, Roijers2015, Yang2019, Belakaria2021, Hayes2022} starting with the conversion of the \gls{PPO} to the multi-objective variant \cite{Khoi2021} for a new baseline. \textcolor{blue}{[4 weeks]}

    \item Investigate the training of policies with constant state space dimension for scalable swarm spacecraft systems \cite{Huttenrauch2018}. \textcolor{blue}{[2 weeks]}
    
    \item Refinement and redaction of final paper(s). \textcolor{blue}{[4 weeks]}
 
    % synthesize the selected gravitational potential model (spherical harmonics) 
    % Assuming an initial gravitational potential models for all bodies with shape models available (1635 asteroids shape models avilable\footnotemark[note1] and 5 comets\footnotemark[note2]). Using the known mass of the bodies (otherwise draw from the distribution of masses of the bodies in the same class), generate a gravitational potential model of each body using the spherical harmonics to a consistent order and degree. \textcolor{blue}{[2 weeks]}
    %       \begin{enumerate}
    %           \item \textcolor{orange}{[week 1/2]} Generate a gravitational potential model for each body with a shape model available using the spherical harmonics method.
    %           \item \textcolor{orange}{[week 1/2]} Generate a gravitational potential model for each body with a shape model available using the mascon method.
    %       \end{enumerate}
    % Using a homogenous polyhedral gravitational potential model from
    
    % according to the model selected and parameters thereof.
    
    % a gravitational potential model (potent) Generate synthetic gravitational potential models according to the selected representation
    % (1635 asteroids shape models avilable\footnotemark[note1])
    
    % This will be form part of the discussion in the thesis, where assumptions will be made about the missing data. 
          
              
    %           across all samples to be modelled by the probability distributions, and the environment models to be used in the simulation environment.
    %           \item \textcolor{orange}{[week 2/2]} Depending on the quantity of data available, select a validation strategy (training, testing and validation data separation). Fit appropriate standard probability distributions to the data and evaluate the goodness-of-fit with respect to the validation samples of the physical characteristics.
    %                 % \item[\[week 2\]] 
    %       \end{enumerate}
    %     %   Document sources, tabulate data, clean data (check for missing data and inconsistencies), and analyse plots for correlations between physical characteristics.
    %       The direction following the results of the above steps will be determined by the results of the goodness-of-fit analysis. If the 
    %       \begin{itemize}
    %         \item \textbf{Case A}: The quantity of samples which contain the set of desired attributes is sufficient to select a subset of samples with an equal representation of all small body orbit classes (i.e. ``Centaur'', ``Inner Main-belt Asteroid'', ``Jupiter Trojan'', ``Halley-type Comet'',...) and composition classes (i.e. ``Carbonaceous'', ``Chondritic'', ``Iron'', ``Silicate'',...) to be used as the training cases for the reinforcement learning agent. No synthetic data will be required and therefore probability density functions will not be generated. If the quantity of samples is more than sufficient (computationally infeasible to train, test and validate the policy's performance across the population), \textbf{stratified sampling will be used to obtain a homogeneous subpopulation (strata) that maintains the representation of the entire population of small bodies in the Solar System.} 
    %         \item \textbf{Case B}: The quantity of samples which contain the set of desired attributes (missing ) is insufficient to select a subset of samples with an equal representation of all small body orbit classes (i.e. ``Centaur'', ``Inner Main-belt Asteroid'', ``Jupiter Trojan'', ``Halley-type Comet'',...) and composition classes (i.e. ``Carbonaceous'', ``Chondritic'', ``Iron'', ``Silicate'',...) to be used as the training cases for the reinforcement learning agent. \textbf{Synthetic data (further samples obtained from probability models) will be required.}
            
    %         % There are a lot of missing data attributes from the desired set across all samples
            
    %         , and the data is not normally distributed. In this case, we will need to use a more complex model to capture the underlying distribution of the data. This could be a mixture of probability distributions, or a more complex distribution such as a Dirichlet distribution. This will be determined by the results of the goodness-of-fit analysis.
    %       \end{itemize}


\end{enumerate}


% and select their resulting environment model representations,
% and the environment models which utilise these characteristics in the dynamical simulation.

