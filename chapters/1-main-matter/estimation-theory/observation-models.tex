\section{Satellite Tracking \& Observation Models}

\subsection{Radar Tracking}

The Deep Space Network (DSN) developed S-band capabilities for spacecraft during
the 1960s.

\begin{itemize}
    \item One-way: The spacecraft generates the downlink signal(s) from an
    onboard oscillator. The DSN compares the received frequency against a
    locally generated frequency
    \item Two-way: The DSN transmits a signal to the spacecraft. The spacecraft
    tracks the phase of the uplink signal and generates a phase coherent
    downlink signal. The DSN compares the received frequency with the same
    reference frequency from which the uplink was generated.
    \item Three-way: The spacecraft is tracked by two stations—one with the
    two-way mode while the other receives and compares the signal to a locally
    generated frequency. The most common application of this mode is during the
    handover between stations at two different Deep Space Communication
    Complexes (DSCCs).
    \item Coherent Three-way: Coherent three-way tracking is three-way tracking
    when the transmitting and receiving stations share a common frequency
    reference. This is possible at all three DSN complexes as all antennas at a
    complex share the same frequency reference.
\end{itemize}

To measure two-way or three-way Doppler shift, the spacecraft must transmit a
downlink signal that is phase coherent with the uplink signal. Table 1 provides the recommended
spacecraft transponder turnaround ratios for various uplink and downlink frequency bands. The
tracking equipment at the DSN 34-m and 70-m stations can accommodate other turnaround
ratios but this support must be negotiated through the JPL Frequency Manager

\begin{table}[htp]
\renewcommand{\arraystretch}{1.5}
\centering
\caption{
    \textbf{Spacecraft Transponder Turnaround Ratios.} K-band (25500-27000 MHz)
    is not listed because the DSN does not support radiometric measurements in
    this band. *Specific integer turnaround ratios must be negotiated through
    the JPL Frequency Manager and must be evaluated within the stated
    range~\cite{Shin2014}.
}
\begin{tabular}{lll}
\hline
\textbf{Uplink} & \textbf{Downlink} & \textbf{Ratio (downlink/uplink)} \\
\hline\hline
S               & S                 & 240/221                          \\
S               & X                 & 880/221                          \\
S               & Ka                & 15.071-15.235*                   \\
X               & S                 & 240/749                          \\
X               & X                 & 880/749                          \\
X               & Ka                & 4.4506-4.4923*                   \\
Ka              & S                 & 0.066959-0.066282*               \\
Ka              & X                 & 0.24561-0.24352*                 \\
Ka              & Ka                & 0.92982-0.93084*                 \\
\hline
\end{tabular}
\end{table}

\begin{table}[htp]
\renewcommand{\arraystretch}{1.5}
\centering
\caption{
    Frequency Bands Allocated by the International Telecommunication Union
    (ITU). Near space bands are used for spacecrafts less than 2 million km from
    Earth, and deep space bands for any distance greater. *Deep Space S-band is
    not available at Madrid tracking stations due to a conflict with IMT- 2000
    users, per agreement between NASA and Secretaria de Estado de
    Telecomunicaciones para la Sociedad de la Informacion (SETSI), January 2001.
    No allocation or not supported by the DSN~\cite{Shin2014}.
}
\begin{tabular}{lllll}
\hline
Band             & \multicolumn{2}{l}{Deep Space Bands} (MHz) & \multicolumn{2}{l}{Near Space Bands} (MHz)\\
                 &                                              Uplink & Downlink &                                             Uplink & Downlink  \\
\hline\hline
          S      &                                        2110–2120* &                    2290–2300 &                                        2025–2110 &                2200–2290 \\
          X      &                                        7145–7190  &                    8400–8450 &                                        7190–7235 &                8450–8500 \\
          K      &                                               **  &                           ** &                                      22550-23150 &              25500–27000 \\
         Ka      &                                      34200–34700  &                  31800–32300 &                                               ** &                       ** \\
\hline
\end{tabular}
\end{table}
\subsubsection{Angle Measurements}

\subsubsection{Ranging}

\subsubsection{Doppler Tracking}

$t_c = t_2 - t_1$

\subsection{Doppler Measurements}

\begin{equation}
    \frac{f_r}{f_t} = \frac{
        1-\mathbf{v}_r\cdot{}\mathbf{e}/c + U_r/c^2 +\mathbf{v}_r^2/(2c^2)
    }{
        1-\mathbf{v}_t\cdot{}\mathbf{e}/c + U_t/c^2 +\mathbf{v}_t^2/(2c^2)
    }
\end{equation}
\begin{equation*}
    \begin{aligned}
        \textrm{where  }
        \mathbf{f}_t, \mathbf{f}_r &= \textrm{the transmitted and received signal frequencies,}\\
        \mathbf{v}_t, \mathbf{v}_t &= \textrm{the transmitters and receivers inertial velocity vectors,}\\
        \mathbf{e} &= \textrm{the unit vector in the direction of the signal propagation,}\\
        c &= \textrm{the speed of light,}\\
        \mathbf{U}_t, \mathbf{U}_r &= \textrm{the Newtonian potential at the trasmitter and receiver.}\\
    \end{aligned}
\end{equation*}

The frequency shift, $f_r/f_t$ cannot be measured instantaneously and must be time averages over
a counting period, $t_c$

\subsubsection{Two-way Range Rate}
\begin{equation}
    \overline{\dot{\rho}}(t) = \frac{c}{2}\frac{(\tau_{2u}+\tau_{2d})-(\tau_{1u}+\tau_{2d})}{t_c} = \frac{1}{2}\frac{(\rho_{2u}+\rho_{2d})-(\rho_{1u} + \rho_{1d})}{t_c},
\end{equation}

\subsubsection{Range}

\subsubsection{One-Way Range Rate}
\begin{equation}
    \overline{\dot{\rho}} = c\frac{(\tau_2-\tau_1)}{t_c}=\frac{(\rho_2-\rho_1)}{t_c}
\end{equation}

\subsubsection{Rational Doppler Bias}
\textcolor{red}{No real need to discuss, cut out in post processing.}
\begin{equation}
    \delta\overline{\dot{\rho}} = \frac{1}{t_c}\int_{t-t_c}^{t}d\cdot{\omega\sin{\alpha}\sin{\omega{t}}\;dt}
\end{equation}

\begin{equation}
    \Delta{\overline{\dot{\rho}}} = \frac{\lambda\omega}{2\pi}\frac{s_R+s_T/T_{1,2}}{2}
\end{equation}



\subsection{Landmark Tracking}

Landmark tracking is facilitated by a perspective projection model, usually a
pin-hole camera model~\cite{Shuang2008} which allows for the mapping of 3D
positions to 2D image points, for simulation and navigational purposes.
Positions in the camera-fixed frame $(x^c, y^c, z^c)$, are mapped to pixel
points on the image plane $(u,v)$ according to the relations:

\begin{equation}
    \begin{aligned}
        u&=f\frac{x^c}{z^c},\\
        v&=f\frac{y^c}{z^c}.
    \end{aligned}
\end{equation}

The line-of-sight vector $b_i^c$ of the $i^{th}$ image point to the camera
exposure center in the camera-fixed frame is defined by:

\begin{equation}
    \mathbf{b}_i^c=\frac{1}{\sqrt{(\bar{x}_i^c)^2+(\bar{y}_i^c)^2+f^2}}
    \begin{bmatrix}
    -\bar{x}^c_i \\
    -\bar{y}^c_i \\
    -f \\
    \end{bmatrix},
\end{equation}

\begin{equation}
    \mathbf{I}_i^I=\frac{1}{\sqrt{(x^I_i - x^I)^2+(y^I_i - y^I)^2+(z^I_i - z^I)^2}}
    \begin{bmatrix}
    x^I_i - x^I \\
    y^I_i - y^I \\
    z^I_i - z^I \\
    \end{bmatrix},
\end{equation}

\begin{equation}
    \tilde{\mathbf{b}}_i^c=\mathbf{A}\mathbf{I}_i^I
\end{equation}


\begin{figure}[htp]
    \centering
    \includegraphics[width=0.55\linewidth]{graphics/landmark_geometry.PNG}
    \caption{
        \textbf{Relative geometrical relationship between spacecraft and landmarks
        for a pin-hole camera model}
    }
\end{figure}





\textcolor{red}{Marie's Thesis + Michael. Model is similar to VLBI model.}

\textcolor{red}{2.1.1 In D.Dirkx for better notation.}

\textcolor{red}{Moyer, formulation for observed and computed quantities in DSN types.}
\subsection{Range Measurements/ Pseudorange}


%\subsection{Doppler Measurements}
%
%\subsubsection{Two-way Range Rate}
%\begin{equation}
%    \overline{\dot{\rho}}(t) = \frac{c}{2}\frac{(\tau_{2u}+\tau_{2d})-(\tau_{1u}+\tau_{2d})}{t_c} = \frac{1}{2}\frac{(\rho_{2u}+\rho_{2d})-(\rho_{1u} + \rho_{1d})}{t_c},
%\end{equation}
%
%\subsubsection{Range}
%
%\subsubsection{One-Way Range Rate}
%\begin{equation}
%    \overline{\dot{\rho}} = c\frac{(\tau_2-\tau_1)}{t_c}=\frac{(\rho_2-\rho_1)}{t_c}
%\end{equation}
%
%\subsubsection{Rational Doppler Bias}
%\textcolor{red}{No real need to discuss, cut out in post processing.}
%\begin{equation}
%    \delta\overline{\dot{\rho}} = \frac{1}{t_c}\int_{t-t_c}^{t}d\cdot{\omega\sin{\alpha}\sin{\omega{t}}\;dt}
%\end{equation}
%
%\begin{equation}
%    \Delta{\overline{\dot{\rho}}} = \frac{\lambda\omega}{2\pi}\frac{s_R+s_T/T_{1,2}}{2}
%\end{equation}
%
%
%\newpage{}
%
%\begin{equation}
%    \rho = ||\mathbf{r}_\Sun - \mathbf{R}_\Earth||
%\end{equation}
%
%
%\begin{equation}
%    \rho = ||\mathbf{r}_\Sun - \mathbf{R}_\Earth||
%\end{equation}
%
%\begin{itemize}
%    \item \textbf{Orbit Determination}
%    \item
%\end{itemize}
%
%\subsection{Rotational State}
%
%\subsection{Gravitational Potential}
%
%\subsection{Shape}
%
%\begin{itemize}
%    \item \textbf{Rotation Model}:
%\end{itemize}