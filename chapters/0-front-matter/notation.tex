\setlength{\glsdescwidth}{0.8\textwidth}
%    \setlength{\glspagelistwidth}{0.1\textwidth}

% \printnoidxglossary[type=symbols, title=Notation, style=long]
\chapter{List of Notations}
% \printnoidxglossary[type=symbols, title={Notation}]
\setglossarysection{subsection}

The notation of this work attempts to provide a clear and concise way of describing the terms and concepts used across the fields of mathematics, computer science, aerospace engineering, and other fields. An example of confusion that can arise is the chosen notation of reinforcement learning, which adapts notation from control theory. The notation of this work is intended to be as clear as possible, and to be as consistent as possible with the rest of the book. In order to do this, suggested notations have been researched and adapted to provide the best possible consistency across the various relevant fields. Where relevant, notation suggestions/guides/references have been provided.

% \renewcommand{\glossarysection}[2][]{{\centering\section*{#2}}}

\printnoidxglossary[
    type={notation:np},
    title={Numbers \& Arrays},
    nonumberlist
]

\printnoidxglossary[
    type={notation:ix},
    title={Indexing},
    nonumberlist
]

\printnoidxglossary[
    type={notation:om},
    title={Orbital Mechanics},
    nonumberlist
]
\printnoidxglossary[
    type={notation:mission},
    title={Other},
    nonumberlist
]
\printnoidxglossary[
    type={notation:set},
    title={Sets \& Graphs},
    nonumberlist
]
\printnoidxglossary[
    type={notation:fn},
    title={Functions},
    nonumberlist
]
\printnoidxglossary[
    type={notation:ml},
    title={Machine Learning},
    nonumberlist
]
\printnoidxglossary[
    type={notation:dl},
    title={Deep Learning},
    nonumberlist
]
\printnoidxglossary[
    type={notation:rl},
    title={Reinforcement Learning},
    nonumberlist
]
\printnoidxglossary[
    type={notation:prob},
    title={Probability and information theory},
]
\setglossarysection{chapter}