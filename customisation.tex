% spell-checker: disable
\usepackage{enumitem}       % Formatted enumerations
\usepackage{graphicx}       % Figure within minipage
\usepackage{wrapfig}        % Wrap text around figure
\usepackage{float}          % For [H] option in figures
\usepackage{color,amsmath}  % Math formatting
\usepackage{mathtools}      % Loads extensions to amsmath (e.g. alignment in matrix environments)
\usepackage{amssymb}        % Extended math fonts (specially Blackboard Bold)
\usepackage{pdfpages}       % to include full pdf pages
\usepackage[utf8]{inputenc} % Extended character catalogue (accents, etc)
\usepackage{color}          % Coloured text
\usepackage{romannum}       % roman numerals
\usepackage{siunitx}
\usepackage{comment}
\usepackage{bm}                 % Allows bold font in equations with \bm
\usepackage{footnotehyper}      % Updated version of \footnote package for hyperref
\usepackage[bottom]{footmisc}   % Ensure footnotes stay at the end of the page
\usepackage{color,soul}         % For highlighting
\usepackage{csquotes}           % For quotations
\usepackage{stackengine}
\usepackage{tocloft}            % Allows adding stuff to the TOC without page numbers
\usepackage{subcaption}         % Enable sub-figures
\usepackage{cleveref}           % Better than autoref


%---------------------------
% Bibliography
%---------------------------
% \usepackage[
%     natbib=true,
%     backend=bibtex,
%     sorting=anyvt,
%     url=true,
%     style=nature,
%     doi=true]{biblatex}

% \addbibresource{library.bib}
%  \usepackage[nottoc]{tocbibind} % Include in TOC
%  \bibliography{library}


%---------------------------
% Table formatting
%---------------------------
\usepackage{array}
\usepackage{makecell}  % Allows for multiline cells
\usepackage{multicol}
\usepackage{multirow}
\usepackage{ctable}
\usepackage{tabu}  % Better customisation than the default tabular
\setlength\tabulinesep{2mm}
\usepackage{booktabs}
\usepackage{dcolumn}  % For aligning equations accross table entries

%---------------------------
% Disable hyphenation
%---------------------------
\tolerance=1
\emergencystretch=\maxdimen
\hyphenpenalty=10000
\hbadness=10000

%---------------------------
% Paper size
%---------------------------
\usepackage{geometry}
\geometry{
	paper=a4paper,  % Change to letterpaper for US letter
	inner=2.5cm,    % Inner margin
	outer=2.5cm,    % Outer margin
% 	bindingoffset=.5cm, % Binding offset
	top=2.2cm,      % Top margin
	bottom=2.2cm,   % Bottom margin
% 	showframe,      % Uncomment to show the precise text area
}

%---------------------------
% Link formatting
%---------------------------
\usepackage{hyperref}
\hypersetup{pdfpagemode={UseOutlines},
    bookmarksopen=true,
    bookmarksopenlevel=0,
    hypertexnames=true,
    colorlinks=true, % Set to false to disable coloring links
    citecolor=[rgb]{0.258,0.38,0.737}, % The color of citations
    linkcolor=black, % The color of references to document elements (sections, figures, etc)
    urlcolor=[rgb]{0.258,0.38,0.737}, % The color of hyperlinks (URLs)
    pdfstartview={FitV},
    breaklinks=true,
}

%-------------------------------
%	Indent multi-line captions
%-------------------------------
% \usepackage{caption}
% \captionsetup{format=hang}

%-------------------------------
%	Code highlighting
%-------------------------------
% Default fixed font does not support bold face
\DeclareFixedFont{\ttb}{T1}{txtt}{bx}{n}{10} % for bold
\DeclareFixedFont{\ttm}{T1}{txtt}{m}{n}{10}  % for normal

% Custom colors
\usepackage{color}
\definecolor{deepblue}{rgb}{0.8,0.47,0.19}
\definecolor{deepred}{rgb}{0.6,0,0}
\definecolor{deepgreen}{rgb}{0,0.5,0}
\definecolor{sky_blue}{rgb}{0.37, 0.55, 0.69}

\usepackage{listings}

% Python style for highlighting
\newcommand\pythonstyle{\lstset{
language=Python,
basicstyle=\ttm,
otherkeywords={self},             % Add keywords here
keywordstyle=\ttb\color{deepblue},
emph={MyClass,__init__},          % Custom highlighting
emphstyle=\ttb\color{deepred},    % Custom highlighting style
stringstyle=\color{deepgreen},
numberstyle=\color{sky_blue},
frame=tb,                         % Any extra options here
showstringspaces=false,            % 
breaklines=true,
postbreak=\mbox{\textcolor{red}{$\hookrightarrow$}\space}
}}


% Python environment
\lstnewenvironment{python}[1][]
{
\pythonstyle
\lstset{#1}
}
{}

% Python for external files
\newcommand\pythonexternal[2][]{{
\pythonstyle
\lstinputlisting[#1]{#2}}}

% Python for external files
% \newcommand\pythonexternal[2][]{{
% \pythonstyle
% \lstinputlisting[#1]{#2}}}

% Python for inline
\newcommand\pythoninline[1]{{\pythonstyle\lstinline!#1!}}


%-------------------------------
%	Math utils
%-------------------------------
% argmin and argmax operators, write \argmin_x or \argmax_x
\DeclareMathOperator*{\argmin}{arg\,min}
\DeclareMathOperator*{\argmax}{arg\,max}

%-------------------------------
%	Nomenclature and formatting
%-------------------------------
\newcommand{\mdollar}[2]{USD #1 million (#2)} % $ <> million (USD)
\newcommand{\bdollar}[2]{USD #1 billion (#2)} % $ <> billion (USD)
 
\newcommand{\valueunit}[2]{\mbox{#1\hspace{0.07cm}#2}}

% Format latin words
\usepackage{lmodern}
\usepackage[T1]{fontenc}
\newcommand\latin[1]{\itshape#1\normalfont}
% \usepackage{times}  % Times font

% Footnote symbols
% The possible styles are:
% arabic    Arabic numerals.
% roman     Lower case Roman numerals.
% Roman     Upper case Roman numerals.
% alph      Alphabetic lower case.
% Alph      Alphabetic upper case.
% fnsymbol  A set of 9 special symbols.
\renewcommand{\thefootnote}{\arabic{footnote}}
