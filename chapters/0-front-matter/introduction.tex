\chapter{Introduction}

Something of paramount importance for the preservation of all known life in the
observable universe, is humanities capabilities to detect, track, and
characterize \glspl{NEO}, a category of small Solar System bodies orbiting the
Sun at roughly 0.98-1.3 times the \textit{Earth-Sun distance} (a distance
roughly an astronomical unit, denoted AU). The term \textit{small body} refers
to any celestial objects which avoided accretion by the Sun, a major planet, or
their largest moons~\cite{Davidsson2021}. Our capabilities of \textit{detection}
and \textit{characterization} exhibit significant potential in their further
pursuit through research and development in our strives towards their complete    % continuous progression / strives
intelligence-based automation.

A momentous failure in automated \textit{detection} was demonstrated by an
event which left scientists {``stunned [and in] true shock"}~\cite{chiu_2019}. A
small body, later named ``2019 OK",  was discovered only \textbf{24 hours} prior
to its closest approach in 2019 when it was approximately 0.01 AU ($\approx$1.5
million kilometers) away from Earth with an apparent magnitude of
14.7~\cite{IAU2019OK} (a measure of relative brightness visible with a 203 mm
telescope aperture~\cite[p.~24]{North2014}). At its closest approach, it passed
Earth at a distance of $\approx$70,000 km, less than one-fifth of the distance  % closest approach or flyby
to the Moon. The object's size was estimated to be between 57-130 m in
diameter~\cite{NASA2019}. The mass of a small body is often estimated by
observing its gravitational interaction with another object, such as another
small body during their close encounter with one another. Without these
observations the mass may not be estimated, however, given the characteristics
of other known small bodies, the effect of the impact of an Asteroid in diameter
range of 2019 OK, would yield between 10-300 megatons of explosive force
\cite{Cellino1999, Rumpf2017}. For reference, 1 megaton is equivalent to 1
million tons of TNT. A next widely known reference: approximately 5 megatons of
energy were released by tsunami waves following the ``2004 Indian Ocean
earthquake" (a.k.a. the ``Sumatra-Andaman earthquake") \cite{Nirupama2006} which
killed at least 225,000 people across a dozen countries. Finally, our largest
reference in the explosive magnitude is the ``Tsar Bomba". This was the
most powerful nuclear weapon ever created by humanity, with an explosive yield
of 50 megatons~\cite{Khan2020} (theoretical yield of 100 megatons),
approximately 1300 times the \textbf{combined yield} of the prominent historical
devices by the names of ``Little Boy" and ``Fat Man", which were nuclear bombs
fueled by highly enriched uranium and dropped on Hiroshima and Nagasaki
respectively, bringing an end to World War II~\cite{osti_1489669}. There is
subsequent consensus amongst scientists on the cause of the failure: objects
approaching in some directions of the sky towards Earth, can exhibit a slow
apparent motion. In the case of similar geometry to ``2019 OK", the apparent
motion of the small bodies were shown to have been as low as 0.1 degrees per
day~\cite{Wainscoat2022}. Generally speaking, the ability to meet temporal
requirements on the characterization of a small body, in the case of a potential
Earth impactor, is contingent on the timing of its detection. The earlier the
detection, the greater the time budget that exists to carry out
characterization, without which we cannot effectively modify an impactor's
trajectory, or modify its structure such that it burns up in Earth's atmosphere.
There have been many proposed ideas for deflecting an asteroid from an
Earth-bound trajectory, however to date, none have been tested in space, and the
feasibility of the proposed methods differ significantly between propositions
\cite{Harris2015}. There is however \glsposs{NASA} \gls{DART}, a mission which
is currently on a 10 month journey to perform a kinetic impact maneuvre into the
secondary, Dimorphos, of the binary system, (65803) Didymos I Dimorphos. This
exercise will be a test of \glsposs{NASA} capability and technologies in
planetary defence, with mission success defined by a change in the binary
orbital period of the system by at least 73 seconds, measured within an accuracy
of 7.3 seconds or less. The mission follows a study performed by the John
Hopkins Applied Physics Labratory with support from members of \gls{NASA} in
2012 \cite{Cheng2012}. As mentioned earlier in the context of ``2019 OK", the
characterization of a small body's mass and structure can be better constrained
through the observation of its interaction with other (small) bodies. This
presents Dimorphos as a particularly attractive target for a deflection maneuver
exercise, considering its close proximity to Didymos, further supported by
excellent Earth-based viewing conditions at the time of the planned impact
(between 2022 September 25 and October 2). The data returned by \gls{DART} will
ultimately determine the momentum transfer efficiency of the kinetic impactor
technique and characterize the resulting effects of the impact~\cite{Cheng2012,
Rivkin2021}.

There have been many missions to small celestial bodies pursuedby all humanity
in our drive to better understand the formation of the Solar System and the
conditions in the early solar nebula.  The major categories of scientific questions, for which small
body exploration bears the promise of answers, are ones addressing the
conditions of the early solar nebular and planetesimal
formation~\cite{Davidsson2021}.


The aim of this work is to elucidate and explore
existing literature defining the boundaries of, and intersections between, the
fields of space exploration and machine autonomy in the context of small body
characterization.




\cite{Rivkin2021}

% our survival: detection, tracking and characterization

% detection failure

% characterization dependent on detection

% characterization: survival, history of the solar system, estimation of economic resources for space missions and bases.



\cite{Marks2022}



% https://ssd.jpl.nasa.gov/tools/sbdb_lookup.html#/?sstr=3843336 database of small bodies from JPL.
% https://cneos.jpl.nasa.gov/nda/overview.html cool NASA app on NEO Deflection calculations.

%
%\begin{fancyquotes}
%    The sciences do not try to explain, they hardly even try to interpret, they
%    mainly make models. By a model is meant a mathematical construct which, with
%    the addition of certain verbal interpretations, describes observed
%    phenomena. The justification of such a mathematical construct is solely and
%    precisely that it is expected to work - that is correctly to describe
%    phenomena from a reasonably wide area. Furthermore, it must satisfy certain
%    aesthetic criteria - that is, in relation to how much it describes, it must
%    be rather simple. - John von Neumann
%\end{fancyquotes}


\section{Brief History of Spacecraft Missions to Small Celestial Bodies}

There have been many missions to small celestial bodies pursued by mankind in
our drive to better understand the formation of the Solar System and the
conditions in the early solar nebula. The term \textit{small body} refers to any
celestial objects which avoided accretion by the Sun, a major planet, or their
largest moons. The major categories of scientific questions which, for which
small body exploration bears promising of answering, are ones addressing
\textit{the conditions of the early solar nebular} and \textit{planetisimanl
formation}~\cite{Davidsson2021}.