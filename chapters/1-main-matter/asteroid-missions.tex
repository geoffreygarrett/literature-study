\section{Historical Asteroid Missions}

The motivations for engaging in space exploration can be broadly divided into two categories: scientific curiosity and national security. In terms of scientific curiosity, humans have long been fascinated by the stars and planets, and the desire to understand our place

There have been many missions to small celestial bodies pursued by mankind in
our drive to better understand the formation of the Solar System and the
conditions in the early solar nebula. The term \textit{small body} refers to any
celestial objects which avoided accretion by the Sun, a major planet, or their
largest moons. The major categories of scientific questions, for which
small body exploration bears promising of answering, are ones addressing
\textit{the conditions of the early solar nebular} and \textit{planetisimanl
    formation}~\cite{Davidsson2021}.

- NEAR Shoemaker on Eros
- NASA's Stardust mission: Annefrank, Wild 2 and Tempel 1
- Japan's Hayabusa Sampling Mission: Itokawa

Hayabusa (aka, MUSES-C) was a Japanese spacecraft designed to return samples
from the near-Earth asteroid Itokawa. It launched on May 9, 2003, and
successfully met up with Itokawa in September 2005. The spacecraft endured
multiple malfunctions during the mission but managed to finish most of its major
objectives. The spacecraft's samples returned to Earth on June 13, 2010, but it
took time for scientists to open its container and check for samples. Hayabusa
mission scientists confirmed in November 2010 that Hayabusa indeed picked up
samples of Itokawa.

Itokawa is a potentially hazardous asteroid that periodically crosses Earth's
orbit; that's one of the reasons this asteroid was chosen for close-up study.
It's about 1,150 feet (350 meters) in diameter and is classified as an S-type
asteroid. Images from the spacecraft showed few impact craters, although a
''rubble pile'' appears on the surface.


- ESA's Rosetta Comet Mission

The European Space Agency's Rosetta spacecraft was a popular mission that
successfully made its way to a comet and landed a probe, called Philae, on the
object's surface. Rosetta launched on March 2, 2004, and made two asteroid
flybys before its last destination: Steins (September 2008) and Lutetia (July
2010).

When Rosetta reached Steins, the probe discovered that the object is a rare
E-type (enstatite) asteroid, meaning that it has iron-poor silicates on its
surface. The asteroid is roughly 4.1 miles (6.6 km) at its longest dimension and
is likely part of a larger object that broke apart. Rosetta spotted impact
craters on Steins' surface, and the space rock's measurements suggest that the
interior consists of rubble. The asteroid will likely disintegrate due to its
delicate interior.

- NASA's Dawn rises

NASA's Dawn mission launched on Sept. 27, 2007, to investigate two large members
of the asteroid belt: Ceres (a dwarf planet) and 4 Vesta (an asteroid). First,
the spacecraft came to Vesta, orbiting the asteroid between July 2011 and
December 2012. Dawn's next and final destination was Ceres, where it entered
orbit on March 6, 2015. (NASA also considered sending Dawn to visit a third
target but ultimately turned down the idea.) The mission ended in late 2018,
when the probe's hydrazine fuel will run out.

- Japan's lost Procyon

Japan's Procyon, also known as Proximate Object Close flyby with Optical
Navigation, launched with Hayabusa2 on Dec. 4, 2014. In 2016, the probe was
supposed to fly by asteroid 2000 DP107, but a problem with the ion-thruster
system on Procyon forced the Japan Aerospace Exploration Agency to abandon the
mission. Procyon did, however, catch a glimpse of Comet 67P, the destination of
the Rosetta mission.

- New Horizons explores Kuiper Belt object Arrokoth

NASA's New Horizons mission was designed to fly past Pluto, which it did in
2015. Because the spacecraft was in fine shape, mission personnel evaluated
other destinations that the probe had fuel to reach and decided to fly past an
object then known only as 2014 MU69, which had been discovered after the
spacecraft launched.

New Horizons' flyby determined that this Kuiper Belt object, now officially
called Arrokoth, was a contact binary, formed when two space rocks glide gently
into each other.


- NASA samples with OSIRIS-REx

NASA's OSIRIS-Rex (Origins, Spectral Interpretation, Resource Identification,
Security, Regolith Explorer) launched on Sept. 8, 2016, en route to Bennu, a
C-type asteroid. The spacecraft arrived at Bennu in August 2018 and spent nearly
two years studying the asteroid from orbit.

On Oct. 20, 2020, the OSIRIS-REx spacecraft conducted the key maneuver of its
mission, capturing a sample of the rocky world to bring back to Earth. The
spacecraft left Bennu in May 2021 and is scheduled to deliver its cargo in
September 2023.


%%%%%%%%%%%%%%%%%%%%%%%%%%%%%%%%%%%%%%%%%%%%%%%%%%%%%%%%%%%%%%%%%%%%%55  FUTURE

- NASA hits an asteroid with a DART
NASA's Double Asteroid Redirection Test (DART) will be a new kind of asteroid
mission for the agency. Instead of focusing on detailed science observations,
DART is a planetary defense mission dedicated to giving scientists their first
real-world data about how they might be able to deflect an asteroid headed for
Earth.

In late 2022, DART will arrive at an asteroid called Didymos, release a cubesat
to record the scene, then slam into Didymos' moon, Dimorphos. Scientists will
watch from Earth to see how much the impact tweaks the moon's orbit around the
larger asteroid.

Later this decade, a European Space Agency mission called Hera will head out to
Didymos as well to study the asteroid and crater after the dust has settled.


- Psyche, a metallic world to explore


In 2022, NASA will launch the Psyche mission to visit an asteroid also called
Psyche. The world is strangely metallic for an asteroid, leaving scientists with
a puzzle — and a hope that the body may turn out to be the bare core of a planet
that lost its rock. The spacecraft will launch in 2022 and reach its target
in 2026, then spend 21 months orbiting Psyche.