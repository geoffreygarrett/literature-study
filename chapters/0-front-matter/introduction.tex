\chapter{Introduction}\label{chap:introduction}

% Context of the research relates to the survival of humanity
Humanity's capabilities to detect, track, and characterise \glspl{NEO} are of paramount importance for the preservation of all known life in the observable universe. \glspl{NEO} are a class of small Solar System bodies orbiting the Sun at roughly 0.98--1.3 times the \textit{Earth-Sun distance}, a distance used in the definition of an astronomical unit (AU). The term \textit{small body} refers to any celestial objects which avoided accretion by the Sun, a major planet, or their largest moons~\cite{Davidsson2021}. Our capabilities of \textit{detection} and \textit{characterisation} of \glspl{NEO} promise significant returns on investment in our progression towards their complete intelligence-based automation. % continuous progression/strives 

% An event that immediately makes clear the importance of small body detection
The case of ``2019 OK'' demonstrated a momentous failure in automated \textit{detection}, leaving scientists ``stunned [and in] true shock''~\cite{chiu_2019}. ``2019 OK'', a small body approaching Earth, was discovered only \textbf{twenty-four hours} before its closest approach in 2019 when it was approximately 0.01 AU (around 1.5 million kilometres) away from Earth with an apparent magnitude of 14.7~\cite{IAU2019OK} (a measure of relative brightness visible with a 203-millimetre telescope aperture~\cite[p.~24]{North2014}). At its closest approach, it passed Earth at a distance of around 70,000 kilometres, less than one-fifth of the Earth-Moon distance. Although the object's size was estimated to be between 57--130 metres in diameter~\cite{NASA2019}, the characterisation of its mass was not feasible due to the lack of observations of the object's gravitational interaction with other objects. However, given the characteristics of other known small bodies, the effect of an Earth impactor comparable to ``2019 OK'' could yield between 10--300 megatons of explosive energy~\cite{Cellino1999, Rumpf2017}. One megaton of explosive energy yield equals that of a million tons of \gls{TNT}. The tsunami waves following the ``2004 Indian Ocean earthquake'' (a.k.a. the ``Sumatra-Andaman earthquake'') released around five megatons of energy resulting in the loss of more than 225,000 lives~\cite{Nirupama2006}. The ``Tsar Bomba'', the most powerful nuclear weapon in history, had an explosive yield of 50 megatons~\cite{Khan2020}. This yield was around 1300 times that of ``Little Boy'' and ``Fat Man'', which were nuclear bombs fueled by highly enriched uranium and dropped on Hiroshima and Nagasaki, respectively, bringing an end to World War II~\cite{malik1985}. Considering that an Earth impactor dimensionally similar to ``2019 OK'' could yield between one-fourth and six times the explosive energy of the ``Tsar Bomba'', it is clear that a period of twenty-four hours between detection and closest approach is an unacceptable failure. There is subsequent consensus amongst scientists on the cause of this failure: objects approaching in some directions of the sky towards Earth can exhibit a slow apparent motion. In case studies representative of ``2019 OK'', the apparent motion was as low as 0.1 degrees per day~\cite{Wainscoat2022}.

Generally speaking, the ability to meet temporal requirements on the characterisation of a small body, in the case of a potential Earth impactor, is contingent on the timing of its detection. The earlier the detection, the greater the time budget to carry out characterisation, without which we cannot confidently modify an impactor's trajectory or structure such that it avoids Earth or burns up in its atmosphere. There have been many proposed ideas for deflecting an asteroid from an Earth-bound trajectory; however, to date, none have been tested in space, and the feasibility across the set of proposed methods exhibits high variance~\cite{Harris2015}. There is, however, \glsposs{NASA} \gls{DART}, a mission that is currently on a 10-month journey to perform a kinetic impact manoeuvre into the secondary, Dimorphos, of the binary system, ``(65803) Didymos I Dimorphos''. This exercise will test \glsposs{NASA} capability and technologies in planetary defence, with mission success defined by a change in the binary orbital period of the system by at least 73 seconds, measured within an accuracy of 7.3 seconds or less. The mission follows a study performed by the John Hopkins Applied Physics Laboratory with support from members of \gls{NASA} in 2012~\cite{Cheng2012}. As mentioned earlier, in the context of ``2019 OK'', the characterisation of a small body's mass, more precisely its physical structure, can be better constrained by observing its interaction with other bodies. Therefore, Dimorphos is a well-suited target for a highly observable deflection manoeuvre exercise, considering its proximity to Didymos, further supported by favourable Earth-based viewing conditions at the time of the planned impact: between September 25\textsuperscript{th} and October 2\textsuperscript{nd} of 2022. The data returned by \gls{DART} will ultimately determine the momentum transfer efficiency of the kinetic impactor technique and characterise the overall effects of the impact~\cite{Cheng2012, Rivkin2021}. The momentum transfer efficiency is ubiquitously given the symbol $\beta$ in the field of asteroid evolution studies and is arguably the most significant factor in attempts to deflect an asteroid from an Earth-bound trajectory. An axiomatic example characteristic of an asteroid, resulting in the variability of $\beta$, is whether it is porous or rocky. It is easier to impart momentum onto a rocky asteroid than a porous one, the latter exhibiting wasted momentum in its deformation rather than its deflection~\cite{Holsapple2012}. Therefore, our efficacy in deflecting an Earth-bound asteroid depends on the degree to which we can characterise its physical structure prior to the manoeuvre.

There have been many missions to small celestial bodies pursued by humanity in our endeavour to understand better the formation of the Solar System and the conditions in the early solar nebula. The major categories of scientific questions, for which small body exploration bears the potential of answering, address the conditions of the early solar nebular and planetesimal formation~\cite{Davidsson2021}. With limited resources for purely scientific missions, optimal characterisation of small bodies is one of many obstacles to the pursuit of understanding the Solar System and the process of planetesimal formation that occurred about 4.5 billion years ago \cite{Klahr2015}. With effective and optimal characterisation techniques of small bodies, the feasibility of in-situ resource acquisition (e.g. “asteroid mining”) for construction, life support systems and propellants also increases significantly. There is no contending that asteroid mining is a complex problem financially, as it involves significant risk at considerable magnitudes of cost, and there is no shortage of doubt as to whether this industry could be profitable within a short time hereafter. Hein et al. \cite{Hein2020} focused their techno-economic analysis on the water supply in space and platinum transported to Earth. They concluded that a key technical parameter for reaching break-even in a shorter time balances the use of smaller but increased numbers of spacecraft per mining mission. The proposed usage of smaller satellites for NEO missions is becoming increasingly popular in published research due to reduced costs and distributed risk of mission failure across multiple satellites as opposed to one~\cite{Wells2006, Laurin2008, Scott2013, Yu2014}. 

This work aims to elucidate and explore existing literature defining the boundaries of, and intersections between, the fields of space exploration and machine autonomy in the context of small body characterisation. This work will focus on how advances in autonomy may enable or enhance the yield of future space missions to small bodies, identifying knowledge gaps in the current understanding of how to fully utilise advances in autonomy for space missions to small bodies. In addition, this work shall contribute to the foundations of knowledge required for developing an architecture that would enable a spacecraft swarm to characterise a small body. The concerned notions of optimality are related to minimising risk and maximised resolution of characteristics. The balance of the optimality between these two factors would be critical in the preparation phase of a kinetic impact manoeuvre for deflecting an Earth-bound object. Risk assessment, planetary defence policy, assignment of the level of urgency, and budgeting operational costs would be key influencers in the tradeoff \cite{Marks2022}. Removed from the context of planetary defence, asteroid mining operations may primarily be driven by the mitigation of mission risk. In contrast, scientific missions in planetary science might opt to trade off slightly higher levels of mission risk if it means exceptional characterisation of a small body which may hold secrets of the early solar nebula or planetesimal formation.

The structure of this work is as follows: first, a brief overview of historic missions in space exploration which involved orbiting or interacting physically with small bodies in \autoref{sec:historical_missions_to_small_bodies}. This is followed by chapter which reviews the literature on machine learning fundamentals (\autoref{sec:fundamentals_of_machine_learning}), deep learning (\autoref{sec:DL}) and reinforcement learning techniques (\autoref{sec:reinforcement_learning}) in \autoref{chap:machine_learning}. Specific attention will be given to applications in autonomous space exploration and characterisation.

\section{Historical Missions to Small Bodies \& Protoplanets}\label{sec:historical_missions_to_small_bodies}

\subsection{Galileo: 951 Gaspra (1991)}

The Galileo mission was launched in 1989 to study Jupiter and its moons. The main challenges faced by the mission were the distance from Earth and the limitations of the onboard instrumentation. However, through careful management of data, Galileo was able to complete its main mission and provide detailed information about the Jupiter system. One of Galileo's early successes was its flyby of asteroid (951) Gaspra in 1991. The images obtained by the spacecraft revealed a body with a very angular surface and a few large craters. Analysis of these images showed that Gaspra is likely an S-type asteroid, meaning that its surface is rich in olivine, pyroxene, and iron-nickel metal - similar to ordinary chondrites and stony-iron meteorites.

\subsection{Galileo: 243 Ida (1993)}

The Galileo spacecraft made its closest approach to asteroid (243) Ida on August 28th, 1993. This was the second asteroid that Galileo had visited, on its way to Jupiter. Observations of (243) Ida by the spacecraft revealed that it is an irregularly shaped object, with dimensions of 59.8 by 25.4 by 18.6 km. The mean radius of Ida is estimated to be 15.7 km, and it has a volume of 16,100 km\textsuperscript{3}. Cratering processes have disturbed the surface of (243) Ida to the point where it has reached equilibrium, meaning that new impacts occur at the same rate as older craters are eroded. This suggests that the asteroid has a substantial regolith, up to 100 m deep. The surface of (243) Ida is also covered in large boulders, which were likely deposited there by impact ejection.

Analysis of data from the Galileo flyby indicated that (243) Ida and other bright members of the Koronis family are S-type asteroids. These asteroids are thought to be relatively young, relative to the age of the solar system.

\subsection{Galileo: Dactyl (1993)}

In 1993, Galileo flew by Ida and discovered that it had a small moon named Dactyl. Dactyl was found to have a diameter of 1.4 km and an orbit that allowed a very precise determination of (243) Ida's mass and density. It was determined that (243) Ida has a low bulk density compared to other asteroids, which is likely due to its moderate to low iron-nickel metal content. The discovery of Dactyl has led to the discovery of many other asteroids with moons, including (45) Eugenia.

\subsection{NEAR-Shoemaker: 253 Mathilde (1996)}

NEAR-Shoemaker was a spacecraft launched by NASA in February 1996 that orbited the asteroid 433 Eros for a year and a half, finally landing on its surface in February 2001. One of its primary goals was to study the asteroid (253) Mathilde. The spacecraft passed by Mathilde on June 27, 1997, taking pictures of its surface at 160-meter resolution. Analysis of these pictures showed that the asteroid has a very dark surface and that its interior is porous and underdense. The most noticeable crater on the asteroid is called Karoo, measuring about 33 kilometres in diameter. This impact crater provides evidence of a very severe collision. These findings were consistent with the asteroid being made of a type of carbonaceous chondrite known as CM chondrites.  NEAR-Shoemaker's observations of (253) Mathilde helped to improve our understanding of the composition and interior structure of this type of asteroid.

\subsection{Deep Space 1: 9969 Braille (1999)}

\Gls{DS1} mission was launched on October 24, 1998 as part of \gls{NASA} Jet Propulsion Laboratory's New Millennium Program. The primary goal of the mission was to validate new technologies, such as the \gls{MICAS}. The \gls{MICAS} was a key scientific instrument carried on board because it demonstrated ultraviolet spectroscopy, short wavelength infrared spectroscopy, and high-resolution imaging capabilities in a single lightweight device. 

On July 29, 1999, one day after its perihelion when it was 1.33 AU from the Sun, the \Gls{DS1} spacecraft flew by the asteroid (9969) Braille at a speed of 15.5 km/s. This encounter with Braille provided a unique opportunity to Probe an Earth-approaching object to further our comprehension of the relationships main belt asteroids share. Scientific observations gathered by MICAS and other instruments during the encounter contributed valuable insights into the properties of Braille and other asteroids \cite{Buratti2004}. 

\subsection{NEAR-Shoemaker: 433 Eros (2000)}

The NEAR-Shoemaker mission to (433) Eros was a success, with the spacecraft landing gently on the asteroid's surface and returning high-resolution images. The images revealed an asteroid that had been altered by space weathering, with few small craters and an abundance of ejecta blocks. The NEAR-Shoemaker spacecraft completed a year of investigating (433) Eros from orbit before landing on the asteroid's surface on February 12, 2001. The landing site was selected so that the spacecraft could maintain continuous Earth contact with the imager pointed at (433) Eros during descent, and 70 descent images were obtained. The pictures were obtained when the spacecraft was as close as 120 m, revealing features as small as 1 cm across. Post-landing analysis indicated a vertical impact velocity of 1.5 to 1.8 m/s and a transverse impact velocity of 0.1\textemdash{}0.3 m/s. The touchdown site was determined to be at 35.78 S, 279.58 W, about 500 m from the nominal site. However, since 2001, there has been a renewed effort to determine the exact location of the landing site using reconstructed pointing information, and as a result, the location of the final landing site has been updated and pinpointed to be in a crater at 41.626$^{\circ}$ South, 80.421$^{\circ}$ East (x = 0.82 $\pm$ 0.01, y = 4.85 $\pm$ 0.01, z = 4.37 $\pm$ 0.01), about 200 m south of the previous estimate (Barnouin et al., 2012).

\subsection{Cassini-Huygens: 2685 Masursky (2000)}

The Cassini-Huygens mission was launched in 1997 toward the Saturnian system to study the planet and its many moons. The spacecraft passed by the asteroid (2685) Masursky on its way to Jupiter and took a series of images of it. The asteroid was found to be about 15-20 km in diameter. The Cassini probe continued to study the Saturn system over the following 13 years until the mission ended with its dive into the planet's atmosphere in 2017.

\subsection{Stardust: 5535 Annefrank (2002)}

The Stardust spacecraft encountered asteroid (5535) Annefrank on November 2, 2002. The encounter was primarily a dress rehearsal and test of operations for the planned Comet Wild 2 encounter. Because Stardust was not equipped with a spectrometer, our understanding of (5535) Annefrank's composition is derived from Earth-based measurements. These measurements place (5535) Annefrank in the S-spectral class, consistent with its orbit within the Flora dynamical class. The instrument suite on Stardust was designed to support the mission objective of collecting a coma sample from Comet Wild 2, so only the \gls{NAVCAM} was used at (5535) Annefrank which led to roughly 40\% of the surface being imaged by Stardust. The images from Stardust were fit by Duxbury et al. to an ellipsoid with diameters of 6.6 $\times$ 5.0 $\times$ 3.4 $\pm$ 2.0 $\times$ 1.0 $\times$ 0.4 km. The main outcome of the (5535) Annefrank encounter is in its comparison with previously collected data. Duxbury et al. (2004) suggested that the object could be composed of a rubble pile, but Stryk and Stooke (2016) argue that lighting on a more coherent (but still irregular) body could provide the appearance of a contact binary (or multiple). Stryk and Stooke suggest (5535) Annefrank resembles ``a miniature Gaspra rather than a large Itokawa.''

\subsection{Hayabusa: 25143 Itokawa (2005)}

The Hayabusa mission to the asteroid (25143) Itokawa was the first ever to return samples from an asteroid. Analysis of the tiny (<10 mm) and larger (30{\textemdash}180 mm) grains that were returned reveals that the surface of Itokawa is rich in olivine-rich minerals, and has been subjected to space weathering processes. These results provide evidence that OCs, the most abundant meteorites found on Earth, come from S-type asteroids. Additionally, the small size of Itokawa (<500 m) and its short lifetime (${\approx}$8 million years) suggest that it may be a reassembled piece of a once larger asteroid. These findings give us important insights into the formation and evolution of asteroids and their potential as resources for future space exploration.

The data collected by Hayabusa showed that Itokawa is an S-type asteroid, meaning that it is made mostly of silicates. The data also showed that Itokawa has been subject to collisions and weathering over its lifetime. The Hayabusa mission has provided valuable information about asteroids and their evolution. It has also shown that S-type asteroids may be a source of chondrites, which are a common type of meteorite that falls to Earth.

Additionally, the small size of Itokawa (<500 m) and its short lifetime (${\approx}$8 million years) suggest that it may be a reassembled piece of a once larger asteroid. These findings give us important insights into the formation and evolution of asteroids and their potential as resources for future space exploration~\cite{Clark2018}. 

\subsection{New Horizons: 132524 APL (2006)}

The Hayabusa spacecraft encountered the asteroid (132524) APL in June 2006. The asteroid was found to be an S-type asteroid with a diameter of 2.3 km. The Ralph instrument, composed of a visible imager \gls{MVIC} and a near-infrared spectrometer \gls{LEISA}, was operating at the time of the encounter and took photographs of the asteroid. The photographs showed that the asteroid was irregularly shaped, but no albedo variations or colour variations could be distinguished on the surface.

\subsection{Rosetta: 2867 Steins (2008)}

The Rosetta spacecraft's flyby of the asteroid (2867) Steins in 2008 provided detailed images and spectroscopic data that revealed it to be a rubble-pile body with a complex surface, likely reshaped by the Yarkovskye O'Keefee Radzievskiie Paddack (YORP) spin-up thermal effect. The spectra obtained by the imaging spectrometer VIRTIS-M, observing from 200{\textemdash}5000 nm, and the camera system, observing with 11 filters in the range 220{\textemdash}960 nm, confirmed that the surface composition is similar to typical E-type asteroids.

\subsection{Rosetta: 21 Lutetia (2010)}

Rosetta's encounter with (21) Lutetia occurred on July 10, 2010. The spacecraft passed within 3100 kilometres of the asteroid and collected data with its scientific instruments. (21) Lutetia has a diameter of approximately 112 kilometres and is a primitive, carbon-rich asteroid. The most significant findings from the encounter were the discovery of landslides on the surface of the asteroid and the detection of a drop in wavelength at 160 nm in its spectra, which has yet to be explained. The data also suggested that the asteroid may be a primordial object which has undergone an early metamorphic or melting process. For more information on Rosetta's encounter with (21) Lutetia, see Barucci et al. (2015) and Weiss et al. (2012).

\subsection{Dawn: 4 Vesta (2011)}

The Dawn spacecraft was launched on September 27, 2007. The primary mission objectives were to study the asteroid (4) Vesta and the dwarf planet (1) Ceres. The Dawn spacecraft entered orbit around (4) Vesta on July 16, 2011, and remained in orbit until September 5, 2012. After departing (4) Vesta, the Dawn spacecraft continued to (1) Ceres, which it entered orbit around on March 6, 2015. The primary mission objectives of studying both asteroids were accomplished. The Dawn mission to (4) Vesta was successful in providing a better understanding of the asteroid's geological history. The impact basins Rheasilvia and Veneneia were discovered, along with the Vestan surface being dominated by a howardite-like composition. The depletion of siderophile elements in \glspl{HED} and models for the composition of the \gls{HED} parent body predicted a core of the size and mass fraction consistent with that inferred from Dawn's gravity data, further strengthening the connection between (4) Vesta and the \glspl{HED}. Additionally, hydrated materials were found in unexpected concentrations on (4) Vesta, indicating that volatile delivery to the inner solar system by primitive bodies was an important process.

\subsection{Chang'e 2: 4179 Toutatis (2012)}

The Chinese lunar orbiter Chang'e 2 flew by the asteroid (4179) Toutatis on December 3, 2012. This was the first time a spacecraft had ever visited this asteroid. The images and data collected by Chang'e 2 have provided valuable information about the shape, spin state, and composition of this asteroid. (4179) Toutatis is an irregularly shaped body which has the appearance of an "uneven peanut." It has a contact binary structure, and there are several craters, concavities, lumps, and boulders on its surface. Authors suggest that (4179) Toutatis is likely a rubble-pile body and its two lobes are possibly formed from contact binaries. The dimensional measurements of (4179) Toutatis were updated by Bu et al. to 4354{$\times$}1835{$\times$}2216{$\pm$}56 m. Jiang et al. (2015) identified more than 200 boulders over the imaged area and used the cumulative boulder size frequency distribution to estimate a surface crater retention age of approximately 1.6{$\pm$}0.3 Gyr.

\subsection{Dawn: 1 Ceres (2015)}

The Dawn spacecraft was launched on September 27, 2007, with the objectives of studying the asteroid Vesta and the dwarf planet (1) Ceres. Dawn discovered several features on (1) Ceres, including the cryovolcano Ahuna Mons, which is the largest mountain on the dwarf planet. The bright streaks on its flanks are thought to be composed of sodium carbonate, and cryovolcanism may also contribute to the formation of other bright deposits on (1) Ceres' surface. Organic material has been found in localized deposits on (1) Ceres, but its origin is still unknown.

\subsection{OSIRIS-REx: 101955 Bennu (2018)}

The ORIRIS-REx spacecraft was launched in September 2016 and flew by Earth in September 2017, finally arriving at (101955) Bennu in November 2018. The spacecraft encountered (101955) Bennu in order to collect a sample from its surface. The spacecraft descended to the surface of (101955) Bennu and used its TAGSAM sampling mechanism to collect between 60{\textemdash}2000 g of material. The collected material will provide information about the early formation time of the solar system and, possibly, about the origin of life on Earth.

\subsection{Hayabusa 2: 162173 Ryugu (2018)}

(162173) Ryugu is a near-Earth C-type asteroid, which is believed to contain organic matter and hydrated minerals. Thus it is expected that a successful sample return may provide fundamental information regarding the origin and evolution of terrestrial planets as well as the origin of water and organics delivered to the Earth. Hayabusa 2 successfully arrived at 162173 Ryugu on June 27, 2018, after 3.5 years of ion engine assisted-interplanetary cruising, and began the asteroid-proximity operation phase. The Hayabusa 2 mission has achieved many successes so far during this phase, including the successful landing of rovers/landers, touchdown and kinetic impact, as well as scientific and engineering assessment conducted synchronously with the spacecraft operations to tackle the unexpectedly harsh environment of Ryugu~\cite{Tsuda2020}. 

\subsection{DART: 65803 Didymos (2022)}

The DART mission is planned for launch in November 2021. The mission will involve crashing a spacecraft into an asteroid in order to change its orbit. The resulting change in the asteroid's orbit will be monitored by ground-based observatories.

The \gls{AIDA} mission is a joint effort by the United States and Europe to investigate the binary asteroid (65803) Didymos and demonstrate asteroid deflection by a kinetic impactor. The Hera mission will study the results of the DART impact in great detail. 

Independently, each mission component: \gls{AIDA}, \gls{DART} and Hera are valuable but synergize greatly towards important knowledge yield in the fields of planetary defence and planetary science. They are associated with challenges that can capture the interest of the public and young generations, which will be key in further work in mitigating the risks of future potential impactors towards celestial bodies inhabited by humanity.

\gls{DART} will be equipped with a camera and a \gls{LIDAR} instrument to measure the size and shape of the impact crater, as well as the distribution of ejected material. Hera will have a suite of cameras, spectrometers, and microphones to study the asteroid's surface before and after the impact \cite{Cheng2012, Rivkin2021}. \gls{AIDA} will be focused on returning data to characterise the momentum transfer efficiency of the kinetic impact manoeuvre and key physical properties of the impacted asteroid. 

\subsection{Lucy: Jupiter trojans (2023)}

Lucy will launch on October 16, 2021. The primary objectives of the mission are to understand the origins of the solar system by studying the Trojan asteroids. Additionally, the mission will also study the main-belt asteroid Donaldjohanson and perform a search for satellites around the Trojan asteroids. The mission is expected to return data in 2033. The spacecraft is named for the fossil of an early human ancestor, Lucy, discovered in 1974, which advanced our understanding of the history of our species. The instruments which will be used on the Lucy spacecraft are a colour camera, infrared imaging spectrometer, high-resolution panchromatic imager, and thermal infrared spectrometer. They will be used to study the surface composition, geology, and bulk properties of the Trojan asteroids. Additionally, the camera will be used to search for satellites around the Trojan asteroids.

The Lucy spacecraft will visit six different Trojan asteroids throughout its mission. The first is Donaldjohanson, which the spacecraft will encounter on 2025 April 20. This encounter is planned as an in-flight demonstration of the spacecraft's capability to reduce mission risk prior to the encounters with the Trojans. Furthermore, Donaldjohanson is a particularly interesting scientific target due to it being a part of the {$\approx$}130{$\pm$} 30 Myr-old Erigone collisional family.

The next five asteroids which Lucy will encounter are all part of Jupiter's L4 Trojan swarm. They are (3548) Eurybates and its satellite Queta, (15094) Polymele, (11351) Leucus, and (21900) Orus. Lucy will study the diversity of these asteroids in order to understand the formation and evolution of the Trojan asteroids. Additionally, data on the surface composition, geology, and bulk properties of these asteroids will be collected in order to understand the connection between the Trojans and other objects in the solar system.

\subsection{Psyche: 16 Psyche (2030)}

The Psyche spacecraft is scheduled to launch in July 2023 at the earliest and will study the asteroid (16) Psyche. Its objectives include understanding the origins of metal-rich asteroids, investigating the effects of a dynamo on a differentiated body, and characterizing the topography and impact crater morphology of (16) Psyche. 

The mission will use high-heritage instruments and radio science to accomplish these objectives. If the magnetometer detects a field, then (16) Psyche is a core and had a core magnetic dynamo that solidified outside-in, allowing the cold solid exterior to record the magnetic field. If very low nickel content is found, with no coherent magnetic field, then scientists may arrive at the hypothesis that (16) Psyche never melted, but consists of highly reduced, primordial metal. The likeliest place for such material to exist is closest to the Sun in the early disk, where temperatures were very hot (reducing) and light elements might have been volatilized away, leaving heavy elements and metals. Three high-heritage instruments will be used: a camera, a spectrometer, and a magnetometer. The mission will also use radio science to map (16) Psyche's gravity field and study the asteroid's topography and impact crater morphology \cite{Hart2018}.


% https://ssd.jpl.nasa.gov/tools/sbdb_lookup.html#/?sstr=3843336 database of small bodies from JPL.
% https://cneos.jpl.nasa.gov/nda/overview.html cool NASA app on NEO Deflection calculations.

%
%\begin{fancyquote}

%\end{fancyquote}

