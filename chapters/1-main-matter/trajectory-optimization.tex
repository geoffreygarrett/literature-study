%%%%%%%%%%%%%%%%%%%%%%%%%%%%%%%%%%%%%%%%%%%%%%%%%%%%%%%%%%%%%%%%%%%%%%%%%%%%%%%%
\chapter{Trajectory Optimization}
%%%%%%%%%%%%%%%%%%%%%%%%%%%%%%%%%%%%%%%%%%%%%%%%%%%%%%%%%%%%%%%%%%%%%%%%%%%%%%%%
Trajectory optimisation is the process of designing a series of states across a
temporal dimension that maximizes (or minimizes) a measure of performance. This
technique is generally used to obtain the open-loop solution of an optimal
control problem within a finite horizon. Due to the common presence of
nonlinearities in the constraint and/or performance functions, trajectory
optimisation problems often employ the use of \textit{nonlinear programming}.

Mathematicaly the problem can be described by $N$ phases, with the independent
variable $t$, where phase $k$ lies within the region
$t_0^{(k)}\leq{t}\leq{t_f^{(k)}}$. The independent variable $t$ very often
describes \text{time} within mosr practical applications, where the phases are
sequential, that is $t_0^{(k+1)}=t_f^{(k)}$, however as stated by
\textit{Betts}, and demononstrated well by \autoref{sec:sundmann}, neither of
these assumptions are required. In phase $k$, the dynamic variables are
described as,
\begin{equation}
  \mathbf{z}=
  \begin{bmatrix}
  \mathbf{x}^{(k)}(t) \\
  \mathbf{u}^{(k)}(t)
  \end{bmatrix},
\end{equation}
where $\mathbf{x}_k$ and $\mathbf{u}_k$ describe the \textit{state variables}
and \textit{control variables} respectively at the $n$ knot points.
bash -ic \usr\bin\pdflatex -file-line-error -interaction=nonstopmode -synctex=1 -output-format=pdf \"-output-directory=/mnt/c/Users/ggarr/OneDrive/Documents/Literature Study/out\" /mnt/c/Users/ggarr/OneDrive/Documents/Literature Study/main.tex

%%%%%%%%%%%%%%%%%%%%%%%%%%%%%%%%%%%%%%%%%%%%%%%%%%%%%%%%%%%%%%%%%%%%%%%%%%%%%%%%
\section{Sims-Flanagan Method}
%%%%%%%%%%%%%%%%%%%%%%%%%%%%%%%%%%%%%%%%%%%%%%%%%%%%%%%%%%%%%%%%%%%%%%%%%%%%%%%%
A common method for low-thrust trajectory optimisation in numerical
astrodynamics, the Sims-Flanagan method \cite{Sims2000}, is the optimisation of
a low-thrust trajectory following transcription into a series of conic arcs
connected by $\Delta{V}$ impulses, as seen in Figure \autoref{fig:sf}.

\begin{figure}[H]
    \centering
    \label{fig:sf}
%    \includegraphics[width=0.4\textwidth]{graphics/1/sims-flanagan-izzo}
    \caption{
        Impulsive $\Delta{V}$ transcription of a low-thrust trajectory, after
        Sims and Flanagan \cite{Yam2010}.
    }
\end{figure}

This method is fast and robust, however due to its discretization of a
continuous low-thrust problem, it can fail to be a faithful representation of
reality. Yam et al. improve upon this method in two ways: 1) the $\Delta{V}$
impulses are replaced with continuous thrust where the low-thrust arcs are
numerically propagated; and 2) the time mesh, or series of knot points, are
optimised together with the trajectory.

%%%%%%%%%%%%%%%%%%%%%%%%%%%%%%%%%%%%%%%%%%%%%%%%%%%%%%%%%%%%%%%%%%%%%%%%%%%%%%%%
\section{Sundmann Transform\label{sec:sundmann}}
%%%%%%%%%%%%%%%%%%%%%%%%%%%%%%%%%%%%%%%%%%%%%%%%%%%%%%%%%%%%%%%%%%%%%%%%%%%%%%%%
Karl Sundman introduced a simple transformation for the time variable, to
regularize the otherwise singular three body problem. A new variable $s$ is
introduced through the relation $ds=dt/r$ which \textit{guarantees an
asymptotically slower flow} near the singularities ~\cite{Sundman1913}.

\begin{equation}
    \begin{aligned}
        r&=\sqrt{x_1 ^2 + x_2 ^ 2 + x_3 ^ 2}\\
        \dot{x}_1 &= x_{4}r\\
        \dot{x}_2 &= x_{5}r\\
        \dot{x}_3 &= x_{6}r\\
        \dot{x}_4 &= -x_{1}/r^2+u_1{r}\\
        \dot{x}_5 &= -x_{2}/r^2+u_2{r}\\
        \dot{x}_6 &= -x_{3}/r^2+u_3{r}\\
        \dot{t} &= r
    \end{aligned}
\end{equation}
Yam et al. demonstrates quite effectively that sampling in the $s$ domain is
useful for trajectory optimization. \cite{Yam2010} The difference between equal
sampling in the $s$ domain compared to that of time, is seen in Figure
\ref{fig:t-s}.

\begin{figure}[H]
    \centering
    \label{fig:t-s}
    \includegraphics[width=0.8\textwidth]{graphics/1/sampling-t-s}
    \caption{
        A trajectory sampled with the same number of a) time spaced segments b)
        s-spaced segments \cite{Yam2010}.
    }
\end{figure}

It is evident from Figure \ref{fig:t-s} that sampling the $s$ domain is superior
in generating a smooth trajectory, in the context of orbital trajectory
discretization. \textcolor{blue}{This presents an interesting aspect when one
considers replacing the temporal variable of time with the s-domain in the
context of an infinite-horizon reinforcement learning problem.}

%%%%%%%%%%%%%%%%%%%%%%%%%%%%%%%%%%%%%%%%%%%%%%%%%%%%%%%%%%%%%%%%%%%%%%%%%%%%%%%%
\section{Optimal Control}
%%%%%%%%%%%%%%%%%%%%%%%%%%%%%%%%%%%%%%%%%%%%%%%%%%%%%%%%%%%%%%%%%%%%%%%%%%%%%%%%

%%%%%%%%%%%%%%%%%%%%%%%%%%%%%%%%%%%%%%%%%%%%%%%%%%%%%%%%%%%%%%%%%%%%%%%%%%%%%%%%
\subsection{Potryagins maximum principal}
%%%%%%%%%%%%%%%%%%%%%%%%%%%%%%%%%%%%%%%%%%%%%%%%%%%%%%%%%%%%%%%%%%%%%%%%%%%%%%%%

%%%%%%%%%%%%%%%%%%%%%%%%%%%%%%%%%%%%%%%%%%%%%%%%%%%%%%%%%%%%%%%%%%%%%%%%%%%%%%%%
\subsection{Model predictive control (MPC)}
%%%%%%%%%%%%%%%%%%%%%%%%%%%%%%%%%%%%%%%%%%%%%%%%%%%%%%%%%%%%%%%%%%%%%%%%%%%%%%%%

%%%%%%%%%%%%%%%%%%%%%%%%%%%%%%%%%%%%%%%%%%%%%%%%%%%%%%%%%%%%%%%%%%%%%%%%%%%%%%%%
\subsection{Recent Research}
%%%%%%%%%%%%%%%%%%%%%%%%%%%%%%%%%%%%%%%%%%%%%%%%%%%%%%%%%%%%%%%%%%%%%%%%%%%%%%%%

%%%%%%%%%%%%%%%%%%%%%%%%%%%%%%%%%%%%%%%%%%%%%%%%%%%%%%%%%%%%%%%%%%%%%%%%%%%%%%%%
\subsubsection{Autoencoder of coordinates}
%%%%%%%%%%%%%%%%%%%%%%%%%%%%%%%%%%%%%%%%%%%%%%%%%%%%%%%%%%%%%%%%%%%%%%%%%%%%%%%%

\url{https://www.youtube.com/watch?v=KmQkDgu-Qp0}
