

% \pgfplotsset{compat=1.17}
% \usetikzlibrary{decorations.pathreplacing}
% \usepackage{pgfplots}

\tikzset{arrowed/.style={decorate,

decoration={show path construction,
moveto code={},
lineto code={
    \draw[#1] (\tikzinputsegmentfirst) --  (\tikzinputsegmentlast);
},
curveto code={},
closepath code={},
}},arrowed/.default={-stealth}}
\pgfplotsset{gradient function/.initial=f,
    dx/.initial=0.01,dy/.initial=0.01}
\pgfmathdeclarefunction{xgrad}{2}{%
    \begingroup%
    \pgfkeys{/pgf/fpu,/pgf/fpu/output format=fixed}%
    \edef\myfun{\pgfkeysvalueof{/pgfplots/gradient function}}%
    \pgfmathparse{(\myfun(#1+\pgfkeysvalueof{/pgfplots/dx},#2)%
    -\myfun(#1,#2))/\pgfkeysvalueof{/pgfplots/dx}}%
    % \pgfmathsetmacro{\mysum}{\mysum+\myfun(\value{isum},#2)}%
    \pgfmathsmuggle\pgfmathresult\endgroup%
}%
\pgfmathdeclarefunction{ygrad}{2}{%
    \begingroup%
    \pgfkeys{/pgf/fpu,/pgf/fpu/output format=fixed}%
    \edef\myfun{\pgfkeysvalueof{/pgfplots/gradient function}}%
    \pgfmathparse{(\myfun(#1,#2+\pgfkeysvalueof{/pgfplots/dy})%
    -\myfun(#1,#2))/\pgfkeysvalueof{/pgfplots/dy}}%
    % \pgfmathsetmacro{\mysum}{\mysum+\myfun(\value{isum},#2)}%
    \pgfmathsmuggle\pgfmathresult\endgroup%
}%

\begin{tikzpicture}
    \begin{axis}
        [width=12cm,%
        declare function={f(\x,\y)=cos(deg(\x)*0.8)*cos(deg(\y)*0.6)*exp(0.1*\x);}]
        \addplot3[surf,shader=interp,domain=-4:4,%samples=81
        ]{f(x,y)};
        \edef\myx{0.15} % first x coordinate
        \edef\myy{-0.15} % first y coordinate
        \edef\mystep{-2}% negative values mean descending
        \pgfmathsetmacro{\myf}{f(\myx,\myy)}
        \edef\lstCoords{(\myx,\myy,\myf)}
        \pgfplotsforeachungrouped\X in{0,...,5}
            {
            \pgfmathsetmacro{\myx}{\myx+\mystep*xgrad(\myx,\myy)}
            \pgfmathsetmacro{\myy}{\myy+\mystep*ygrad(\myx,\myy)}
            \pgfmathsetmacro{\myf}{f(\myx,\myy)}
            \edef\lstCoords{\lstCoords\space (\myx,\myy,\myf)}
        }
        \addplot3[samples y=0,arrowed] coordinates \lstCoords;
    \end{axis}
\end{tikzpicture}