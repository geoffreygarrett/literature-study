\section{Linearization}

Given the complexity of the models which depend on a variety of
parameters which are used in the description of the dynamic motion or
measurement process, it is customary to linearize the relations between
observables and the independent parameters in order to obtain expressions which
are more easily handled.

\subsection{Taxonomy of Partials}
\subsubsection{The State Transition Matrix}

Let the state vector $\mathbf{y(t)}$ of the dynamic system at an arbitrary epoch
$t$ be the concatenation of the position $\mathbf{r}(t)$ and velocity vector
$\mathbf{v}(t)$. A state vector at a specified epoch is given by
$\mathbf{y}(t_0)$. The state transition matrix $\bm{\Phi}(t,t_0)$ describes the
change of the state vector at a time $t$ due to a change in the initial state
vector at time $t_0$

\begin{equation}
    \bigg(\frac{\partial\mathbf{y}(t)}{\partial{\mathbf{y}(t_0)}}\bigg)_{6\times{6}}=\bm{\Phi}(t, t_0).
    \label{eq:linear_stm}
\end{equation}

\subsubsection{The Sensitivity Matrix}

For a given set of $n_p$ parameters which are dependent variables in
the calculation of acceleration acting on on the system of bodies $\mathbf{p}$,
their dependence is described by the sensitivity matrix, i.e. the partial
derivatives

\begin{equation}
    \bigg(\frac{\partial\mathbf{y}(t)}{\partial{\mathbf{p}}}\bigg)_{6\times{n_p}}=\bm{S}(t)
\end{equation}

with respect to the force model parameters. The parameters $p$ often include
coefficients which are empirical in nature, these include but are not limited to
the coefficients of lift, drag, and radiation pressure.

\subsubsection{Partials of measurements with respect to the state vector}
The linearized dependence of 

\subsection{Solar Radiation Pressure}

\subsubsection{Partial derivative with respect to state}

\begin{equation}
    \frac{\partial{\bm{a}_{s}}}{\partial{\mathbf{r}}} =
    -P_\Sun\frac{A}{m}\frac{AU^2}{r^5}
    \begin{bmatrix}
    3r_x^2-r^2 & 3r_xr_y    & 3r_xr_z \\
    3r_yr_x    & 3r_y^2-r^2 & 3r_yr_z \\
    3r_zr_x    & 3r_zr_y    & 3r_z^2-r^2 \\
    \end{bmatrix}
    \label{eq:partial_srp_state}
\end{equation}

\subsubsection{Partial derivative with respect to coefficient of radiation pressure}

\begin{equation}
    \frac{\partial\bm{a}_{s}}{\partial{C_r}}
    =
    \frac{1}{C_R}\bm{a}_s
    =
    -P_\Sun\frac{A}{m}\frac{\mathbf{r}}{r^3}AU^2
    \label{eq:partial_srp_cr}
\end{equation}

\subsection{Thrust}

\begin{equation}
    \frac{\bm{a}_T}{\partial{}}
\end{equation}

\subsection{Gravitational Potential: Point Mass}

\subsubsection{Partial derivative with respect to state}
\begin{equation}
    \frac{\partial{\bm{a}_{g}}}{\partial{\mathbf{r}}}  =
    \frac{\mu}{r^5}
    \begin{bmatrix}
    3r_x^2-r^2 & 3r_xr_y    & 3r_xr_z \\
    3r_yr_x    & 3r_y^2-r^2 & 3r_yr_z \\
    3r_zr_x    & 3r_zr_y    & 3r_z^2-r^2 \\
    \end{bmatrix}
    \label{eq:partial_point_mass_wrt_state}
\end{equation}

\subsubsection{Partial derivative with respect to gravitational parameter}
\begin{equation}
    \frac{\partial{\bm{a}_{g}}}{\partial{\mu}} =
    \frac{1}{\mu}\bm{a}_g =
    -\frac{\mathbf{r}}{r^3}
    \label{eq:partial_point_mass_wrt_G}
\end{equation}

%%%%%%%%%%%%%%%%%%%%%%%%%%%%
\subsection{Gravitational Potential: Tri-axial ellipsoid}
\subsubsection{Partial derivative with respect to state}
\begin{equation}
    \frac{\partial{\bm{a}_{g}}}{\partial{\mathbf{r}}} =
    \mu 
    \begin{bmatrix}
    
    \end{bmatrix}
    \label{eq:partial_triaxial_wrt_state}
\end{equation}

\subsubsection{Partial derivative with respect to characteristic parameters}
\begin{equation}
    \mathbf{p}_{g}=
    \begin{bmatrix}
    \rho & a & b & c \\
    \end{bmatrix}^T
\end{equation}

\begin{equation}
    j_1=
\end{equation}

\begin{equation}
    \frac{\partial{\bm{a}}_g}{\partial{\rho}} = 
    G\frac{4}{3}\pi
    \begin{bmatrix}
    {abc}{r_x}{R_D(b^2 + \kappa_0, c^2 + \kappa_0, a^2 + \kappa_0)}\\
    {abc}{r_y}{R_D(a^2 + \kappa_0, c^2 + \kappa_0, b^2 + \kappa_0)}\\
    {abc}{r_z}{R_D(a^2 + \kappa_0, b^2 + \kappa_0, c^2 + \kappa_0)}\\
    \end{bmatrix}
\end{equation}

\begin{equation}
    \frac{\partial{\bm{a}}_g}{\partial{\mathbf{p}_g}} = 
    G\frac{4}{3}\pi
    \begin{bmatrix}
    {abc}{r_x}{R_{D_x}} & \rho bc({r_x}{R_{D_x}} + a\frac{\partial{R_{D_x}}}{\partial{a}}) & \rho ac({r_x}{R_{D_x}} + b\frac{\partial{R_{D_x}}}{\partial{b}}) & \rho ab({r_x}{R_{D_x}} + c\frac{\partial{R_{D_x}}}{\partial{c}})\\
    {abc}{r_y}{R_{D_y}} & \rho bc({r_y}{R_{D_y}} + a\frac{\partial{R_{D_y}}}{\partial{a}}) & \rho ac({r_y}{R_{D_y}} + b\frac{\partial{R_{D_y}}}{\partial{b}}) & \rho ab({r_y}{R_{D_y}} + c\frac{\partial{R_{D_y}}}{\partial{c}})\\
    {abc}{r_z}{R_{D_z}} & \rho bc({r_z}{R_{D_z}} + a\frac{\partial{R_{D_z}}}{\partial{a}}) & \rho ac({r_z}{R_{D_z}} + b\frac{\partial{R_{D_z}}}{\partial{b}}) & \rho ab({r_z}{R_{D_z}} + c\frac{\partial{R_{D_z}}}{\partial{c}})\\
    \end{bmatrix}
\end{equation}

\subsection{Gravitational Potential: Expansion in spherical harmonics}
