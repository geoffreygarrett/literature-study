\chapter{Introduction}

One component of paramount importance for the preservation of all known life in
the observable universe, is humanities capabilities to detect, track, and
characterize \glspl{NEO}. Two of these aspects: \textit{detection} and
\textit{characterization}, exhibit significant potential returns in our
continued progression towards their complete intelligence-based automation.

A significant failure in automated \textit{detection} was demonstrated by an
event which left scientists {``stunned [and in] true shock"}~\cite{chiu_2019}. A
small body, later named ``2019 OK",  was discovered only \textbf{24 hours} prior
to its closest approach in 2019 when it was approximately 0.01 AU ($\approx$1.5
million kilometers) away from Earth with an apparent magnitude of
14.7~\cite{IAU2019OK} (a measure of relative brightness visible with a 203 mm
telescope aperture~\cite[p.~24]{North2014}). It passed Earth at a distance of
$\approx$70,000 km, less than one-fifth of the distance to the Moon. The % closest approach or flyby
 object's size was estimated to be between 57-130 m in
diameter~\cite{NASA2019}. The mass of a small body is often estimated by
observing its gravitational interaction with another object, such as another
small body during their close encounter with one another. Without these
observations the mass may not be estimated, however, given the characteristics
of other known small bodies, the effect of the impact of an Asteroid in diameter
range of 2019 OK, would yield between 10-300 megatons of explosive force
\cite{Cellino1999, Rumpf2017}. For reference, 1 megaton is equivalent to 1
million tons of TNT. A next widely known reference: approximately 5 megatons of
energy were released by tsunami waves following the ``2004 Indian Ocean
earthquake" (a.k.a. the ``Sumatra-Andaman earthquake") \cite{Nirupama2006} which
killed at least 225,000 people across a dozen countries. Finally, our largest
reference in the magnitude of explosive yield, the ``Tsar Bomba". This was the
most powerful nuclear weapon ever created by humanity, with an explosive yield
of 50 megatons~\cite{Khan2020} (theoretical yield of 100 megatons),
approximately 1300 times the \textbf{combined yield} of the prominent historical
devices by the names of ``Little Boy" and ``Fat Man", which were nuclear bombs
fueled by highly enriched uranium and dropped on Hiroshima and Nagasaki
respectively, bringing an end to World War II~\cite{osti_1489669}.
%It is in our
%best interest to remember our encounters with calamity throughout history, so
%that we may attempt to foresee the true consequences that an impact event would
%have on Earth, and ultimately preserve the future of humanity.
There is consensus amongst scientists on the cause of the failure: objects
approaching in some directions of the sky towards Earth, can exhibit a slow
apparent motion.


\cite{Marks2022}


There have been many missions to small celestial bodies pursued by all humanity
in our drive to better understand the formation of the Solar System and the
conditions in the early solar nebula. The term \textit{small body} refers to any
celestial objects which avoided accretion by the Sun, a major planet, or their
largest moons. The major categories of scientific questions, for which
small body exploration bears the promise of answers, are ones addressing
\textit{the conditions of the early solar nebular} and \textit{planetesimal
formation}~\cite{Davidsson2021}.


% https://ssd.jpl.nasa.gov/tools/sbdb_lookup.html#/?sstr=3843336 database of small bodies from JPL.
% https://cneos.jpl.nasa.gov/nda/overview.html cool NASA app on NEO Deflection calculations.

%
%\begin{fancyquotes}
%    The sciences do not try to explain, they hardly even try to interpret, they
%    mainly make models. By a model is meant a mathematical construct which, with
%    the addition of certain verbal interpretations, describes observed
%    phenomena. The justification of such a mathematical construct is solely and
%    precisely that it is expected to work - that is correctly to describe
%    phenomena from a reasonably wide area. Furthermore, it must satisfy certain
%    aesthetic criteria - that is, in relation to how much it describes, it must
%    be rather simple. - John von Neumann
%\end{fancyquotes}

\section{Brief History of Spacecraft Missions to Small Celestial Bodies}

There have been many missions to small celestial bodies pursued by mankind in
our drive to better understand the formation of the Solar System and the
conditions in the early solar nebula. The term \textit{small body} refers to any
celestial objects which avoided accretion by the Sun, a major planet, or their
largest moons. The major categories of scientific questions which, for which
small body exploration bears promising of answering, are ones addressing
\textit{the conditions of the early solar nebular} and \textit{planetisimanl
formation}~\cite{Davidsson2021}.