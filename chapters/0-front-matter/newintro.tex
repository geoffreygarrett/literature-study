\chapter{Introduction}

% Context of the research relates to the survival of humanity
Something of paramount importance for the preservation of all known life in the observable universe is humanities capabilities to detect, track, and characterize \glspl{NEO}, a category of small Solar System bodies orbiting the Sun at roughly 0.98-1.3 times the \textit{Earth-Sun distance} (a distance roughly an astronomical unit, denoted AU). The term \textit{small body} refers to any celestial objects which avoided accretion by the Sun, a major planet, or their largest moons~\cite{Davidsson2021}. Our capabilities of \textit{detection} and \textit{characterization} exhibit significant potential in their further pursuit through research and development in our strives towards complete intelligence-based automation. % continuous progression/strives

% An event that immediately makes clear the importance of small body detection
A momentous failure in automated \textit{detection} was demonstrated by an event that left scientists {``stunned [and in] true shock"}~\cite{chiu_2019}. A small body, later named ``2019 OK",  was discovered only \textbf{24 hours} before its closest approach in 2019 when it was approximately 0.01 AU ($\approx$1.5 million kilometres) away from Earth with an apparent magnitude of 14.7~\cite{IAU2019OK} (a measure of relative brightness visible with a 203 mm telescope aperture~\cite[p.~24]{North2014}). At its closest approach, it passed \comment{closest approach or flyby} Earth at a distance of $\approx$70,000 km, less than one-fifth of the distance to the Moon. The object's size was estimated to be between 57-130 m in diameter~\cite{NASA2019}. The mass of a small body is often estimated by observing its gravitational interaction with another object, such as another small body, during their close encounter with one another. Without these observations, the mass may not be estimated, however, given the characteristics of other known small bodies, the effect of the impact of an Asteroid comparable to 2019 OK would yield between 10-300 megatons of explosive force \cite{Cellino1999, Rumpf2017}. For reference, 1 megaton is equivalent to 1 million tons of TNT. A following widely known reference: approximately 5 megatons of energy were released by tsunami waves following the ``2004 Indian Ocean earthquake" (a.k.a. the ``Sumatra-Andaman earthquake") \cite{Nirupama2006} which killed at least 225,000 people across a dozen countries. Finally, our most significant reference in explosive magnitude is the ``Tsar Bomba". This was the most powerful nuclear weapon ever created by humanity, with an explosive yield of 50 megatons~\cite{Khan2020} (theoretical yield of 100 megatons), approximately 1300 times the \textbf{combined yield} of the prominent historical devices by the names of ``Little Boy" and ``Fat Man", which were nuclear bombs fueled by highly enriched uranium and dropped on Hiroshima and Nagasaki respectively, bringing an end to World War II~\cite{malik1985}. There is subsequent consensus amongst scientists on the cause of the failure: objects approaching in some directions of the sky towards Earth can exhibit a slow apparent motion. In the case of similar geometry to ``2019 OK"s approach, the apparent motion of comparable small bodies was shown to have been as low as 0.1 degrees per day~\cite{Wainscoat2022}. Generally speaking, the ability to meet temporal requirements on the characterization of a small body, in the case of a potential Earth impactor, is contingent on the timing of its detection. The earlier the detection, the greater the time budget to carry out characterization, without which we cannot effectively modify an impactor's trajectory or structure such that it burns up in Earth's atmosphere. There have been many proposed ideas for deflecting an asteroid from an Earth-bound trajectory; however, to date, none have been tested in space, and the feasibility of the proposed methods differs significantly~\cite{Harris2015}. There is, however, \glsposs{NASA} \gls{DART}, a mission that is currently on a 10-month journey to perform a kinetic impact manoeuvre into the secondary, Dimorphos, of the binary system, ``(65803) Didymos I Dimorphos". This exercise will test \glsposs{NASA} capability and technologies in planetary defence, with mission success defined by a change in the binary orbital period of the system by at least 73 seconds, measured within an accuracy of 7.3 seconds or less. The mission follows a study performed by the John Hopkins Applied Physics Laboratory with support from members of \gls{NASA} in 2012~\cite{Cheng2012}. As mentioned earlier in the context of ``2019 OK", the characterization of a small body's mass and structure can be better constrained by observing its interaction with other (small) bodies. This presents Dimorphos as an attractive target for a deflection manoeuvre exercise, considering its proximity to Didymos, further supported by excellent Earth-based viewing conditions at the time of the planned impact (between 2022 September 25$^{\text{th}}$ and October 2$^{\text{nd}}$). The data returned by \gls{DART} will ultimately determine the momentum transfer efficiency of the kinetic impactor technique and characterize the effects of the impact~\cite{Cheng2012, Rivkin2021}.

There have been many missions to small celestial bodies pursued by humanity in our endeavour to understand better the formation of the Solar System and the conditions in the early solar nebula. The major categories of scientific questions, for which small body exploration bears the potential of answering, address the conditions of the early solar nebular and planetesimal formation~\cite{Davidsson2021}. With limited resources for purely scientific missions, optimal characterization of small bodies is one of many obstacles to understanding the Solar System and the process of "planetesimal formation" that occurred about 4.5 billion years ago~\cite{Klahr2015}. With effective and optimal characterization techniques of small bodies, the feasibility of in-situ resource acquisition (e.g. ``asteroid mining") for construction, life support systems and propellants also increases significantly. There is no contending that asteroid mining is a simple problem financially, it has significant cost implications, and there is no shortage of doubt as to whether this industry could be made profitable soon. Hein et al. focused their techno-economic analysis on the water supply in space and the collection and transport of platinum back to Earth and concluded that a key technical parameter for reaching break-even in a shorter period is the use of smaller but increased numbers of spacecraft per mission ~\cite{Hein2020}. The proposed use of smaller satellites for \gls{NEO} missions is becoming increasingly popular in published research due to decreased costs and distributed risk of mission failure across multiple satellites as opposed to one~\cite{Wells2006, Laurin2008, Scott2013, Yu2014}.
