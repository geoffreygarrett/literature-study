% MIT License
%
% Copyright (c) 2021 Geoffrey H. Garrett
%
% Permission is hereby granted, free of charge, to any person obtaining a copy
% of this software and associated documentation files (the "Software"), to deal
% in the Software without restriction, including without limitation the rights
% to use, copy, modify, merge, publish, distribute, sublicense, and/or sell
% copies of the Software, and to permit persons to whom the Software is
% furnished to do so, subject to the following conditions:
%
% The above copyright notice and this permission notice shall be included in all
% copies or substantial portions of the Software.
%
% THE SOFTWARE IS PROVIDED "AS IS", WITHOUT WARRANTY OF ANY KIND, EXPRESS OR
% IMPLIED, INCLUDING BUT NOT LIMITED TO THE WARRANTIES OF MERCHANTABILITY,
% FITNESS FOR A PARTICULAR PURPOSE AND NONINFRINGEMENT. IN NO EVENT SHALL THE
% AUTHORS OR COPYRIGHT HOLDERS BE LIABLE FOR ANY CLAIM, DAMAGES OR OTHER
% LIABILITY, WHETHER IN AN ACTION OF CONTRACT, TORT OR OTHERWISE, ARISING FROM,
% OUT OF OR IN CONNECTION WITH THE SOFTWARE OR THE USE OR OTHER DEALINGS IN THE
% SOFTWARE.

%%%%%%%%%%%%%%%%%%%%%%%%%%%%%%%%%%%%%%%%%%%%%%%%%%%%%%%%%%%%%%%%%%%%%%%%%%%%%%%
% ACKNOWLEDGEMENTS
%%%%%%%%%%%%%%%%%%%%%%%%%%%%%%%%%%%%%%%%%%%%%%%%%%%%%%%%%%%%%%%%%%%%%%%%%%%%%%%
% Design and implementation of this diagram was inspired and adapted from:
% https://tex.stackexchange.com/questions/104334/tikz-diagram-of-a-perceptron

%%%%%%%%%%%%%%%%%%%%%%%%%%%%%%%%%%%%%%%%%%%%%%%%%%%%%%%%%%%%%%%%%%%%%%%%%%%%%%%
% DEPENDENCIES
%%%%%%%%%%%%%%%%%%%%%%%%%%%%%%%%%%%%%%%%%%%%%%%%%%%%%%%%%%%%%%%%%%%%%%%%%%%%%%%
%\usepackage{tikz}
%\usetikzlibrary{decorations.pathreplacing}    % for TikZ braces
%\usetikzlibrary{positioning}                  % for TikZ relative positioning

%%%%%%%%%%%%%%%%%%%%%%%%%%%%%%%%%%%%%%%%%%%%%%%%%%%%%%%%%%%%%%%%%%%%%%%%%%%%%%%
% USER STYLING
%%%%%%%%%%%%%%%%%%%%%%%%%%%%%%%%%%%%%%%%%%%%%%%%%%%%%%%%%%%%%%%%%%%%%%%%%%%%%%%

% TikZ node design.
\tikzset{basic/.style={draw,text width=1em,text badly centered}}
\tikzset{input/.style={}}
\tikzset{output/.style={}}
\tikzset{weight/.style={basic,circle}}
\tikzset{hidden/.style={basic,circle}}
\tikzset{function/.style={basic,circle}}
\def\layersep{3em}
\def\transferx{9em}
\def\hiddenx{7em}
\def\hiddenxn{12em}

% Labels and symbols.
\def\activationlabel{threshold step}  % activation function label
\def\activationsymbol{$H$}              % activation function symbol
\def\transferlabel{sum}      % transfer function label
\def\transfersymbol{$\sum$}                % transfer function symbol
\def\outputsymbol{$\text{NOR}(x_1, x_2)$}                     % output symbol
\def\inputsymbol{$x$}                      % input symbol
\def\inputvecsymbol{$\mathbf{x}$}          % input vector symbol
\def\weightslabel{weights}                 % input vector symbol
\def\biassymbol{$b$}                       % bias symbol

\def\sep{4em}
\def\L{\gls{L}}       % number of hidden layers
\def\y{\gls{y_true}}  % output vector
\def\x{\gls{ml:x}}    % input vector
\def\h{\gls{a_vec}}   % hidden output
\def\nx{\gls{ml:n_x}}   % hidden output
\def\ny{\gls{ml:n_y}}   % hidden output
\def\z{\gls{dl:z}}   % hidden output
\def\a{\gls{dl:a}}   % hidden output

%%%%%%%%%%%%%%%%%%%%%%%%%%%%%%%%%%%%%%%%%%%%%%%%%%%%%%%%%%%%%%%%%%%%%%%%%%%%%%%
% TIKZ PICTURE
%%%%%%%%%%%%%%%%%%%%%%%%%%%%%%%%%%%%%%%%%%%%%%%%%%%%%%%%%%%%%%%%%%%%%%%%%%%%%%%
\begin{tikzpicture}

    \node[function] at ({\transferx}, -{int(2)})  (transfer) {\transfersymbol};
    \node[function, right=3em of transfer] (activation) {\activationsymbol};
    \node[below of=activation,font=\scriptsize,text width=3em] {\activationlabel};
    \node[below of=transfer,font=\scriptsize,text width=3em] {\transferlabel};
    \node[output, right=\layersep of activation] (output) {\outputsymbol};
    \path[draw,->] (transfer) -- node[above] {\z} (activation);
    \path[draw,->] (activation) -- node[above] {\a} (output);

    \node[input] at (0, -1) (X-1) {$1$};
    \def\wone{$1$}
    \node[weight, label={[xshift=-0.0em]center:\wone}] at (\layersep, -1) (W-1) {\phantom{\wone}};

    \node[input] at (0, -2) (X-2) {$x_1$};
    \def\wtwo{$-1$}
    \node[weight, label={[xshift=-0.0em]center:\wtwo}] at (\layersep, -2) (W-2) {\phantom{\wtwo}};

    \node[input] at (0, -3) (X-3) {$x_2$};
    \def\wthree{$-1$}
    \node[weight, label={[xshift=-0.0em]center:\wthree}] at (\layersep, -3) (W-3) {\phantom{\wthree}};

    \path[draw,->] (X-1) -- (W-1);
    \path[draw,->] (X-2) -- (W-2);
    \path[draw,->] (X-3) -- (W-3);
    \path[draw,->] (W-1) -- (transfer);
    \path[draw,->] (W-2) -- (transfer);
    \path[draw,->] (W-3) -- (transfer);

    % Brace for bias.
    \node[left=1em of X-1] (bias-brace) {};
    \node[above=1em of bias-brace] (bias-brace-up) {};
    \node[below=1em of bias-brace] (bias-brace-down) {};
    \draw[decorate,decoration = {brace}] (bias-brace-down) --  (bias-brace-up);
    \node[left of=bias-brace,font=\scriptsize] {bias, \biassymbol};

    % Brace for input.
    \node[below=4.5em of bias-brace-down] (input-brace-down) {};
    \node[below=2.25em of bias-brace-down] (input-brace) {};
    \draw[decorate,decoration = {brace}] (input-brace-down) --  (bias-brace-down);
    \node[left of=input-brace,font=\scriptsize] {inputs, \inputvecsymbol};

    % Weights label.
    \node[below of=W-3,font=\scriptsize] {\weightslabel};

\end{tikzpicture}
