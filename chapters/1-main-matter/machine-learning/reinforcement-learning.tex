\section{Reinforcement Learning\label{ssec:RL}}
%%%%%%%%%%%%%%%%%%%%%%%%%%%%%%%%%%%%%%%%%%%%%%%%%%%%%%%%%%%%%%%%%%%%%%%%%%%%%%%%

\subsection{Key Concepts in Reinforcement Learning}

\subsubsection{Markov Decision Processes (MDP)}

So far, we’ve discussed the agent’s environment in an informal way, but if you
try to go digging through the literature, you’re likely to run into the standard
mathematical formalism for this setting: Markov Decision Processes (MDPs). An
%MDP is a 5-tuple, \langle S, A, R, P, \rho_0 \rangle, where

S is the set of all valid states,
A is the set of all valid actions,
R : $S \times A \times S \to \gls{set:R}$ is the reward function, with $r_t = R(s_t, a_t, s_{t+1})$,
P : $S \times A \to \mathcal{P}(S)$ is the transition probability function, with
$P(s'|s,a)$ being the probability of transitioning into state $s'$ if you start
in state $s$ and take action $a$,
and $\rho_0$ is the starting state distribution.

The name Markov Decision Process refers to the fact that the system obeys the
Markov property: transitions only depend on the most recent state and action,
and no prior history.

\tikzset{basic/.style={draw,text width=1em,text badly centered}}
\tikzset{component/.style={rectangle, draw=black, thick, text width=7em,align=center, rounded corners, minimum height=2em}}

\begin{figure}[!htp]
    \centering
    \captionsetup{format=hang} % hanging captions

    \tikzstyle{block} = [rectangle, draw,
    text width=8em, text centered, rounded corners, minimum height=4em]

    \tikzstyle{line} = [draw, -latex]

    \begin{tikzpicture}[node distance = 6em, auto, thick]
        \node [block] (Agent) {Agent};
        \node [block, below=5em of Agent] (Environment) {Environment};
        \node [left=3em of Environment] (Dashed) {};
        \node [above=1.7em of Dashed] (Dashed-up) {};
        \node [below=1.7em of Dashed] (Dashed-down) {};

        \path [line] (Agent.0) --++ (4em,0em) |- node [near start]{Action, $a_t$} (Environment.0);

        \path [line] (Environment.190) -- (Environment.190-|Dashed-down.east) node [midway, label={[xshift=-0.0em]center:$s_{t+1}$}] {\phantom{$s_{t+1}$}};
        \path [line] (Environment.170) -- (Environment.170-|Dashed-up.east) node [midway, above, label={[xshift=-0.0em]center:$r_{t+1}$}] {\phantom{$r_{t+1}$}};

        \path [line] (Environment.190|-Environment.190) --++ (-6em,0em) |- node [near start] {} node [near start, label={[xshift=-0.0em]center:State, $s_t$}] {\phantom{State, $s_t$}} (Agent.170);
        \path [line] (Environment.170|-Environment.170) --++ (-4.25em,0em) |- node [near start, right] {} node [near start, right, label={[xshift=-0.0em]center:Reward, $r_t$}] {\phantom{Reward, $r_t$}} (Agent.190);

        \draw[dotted] (Dashed-up.east) -- (Dashed-down.east);


    \end{tikzpicture}
    \caption{\textbf{The agent-environment interaction interface} illustrates the interaction between an agent and its environment. The agent takes an action $a_t$ and receives a reward $r_{t+1}$ and a new state $s_{t+1}$, subject to the environment's dynamics.}
    \label{fig:agent-environment}

\end{figure}

\begin{itemize}
    \item states and observations,
    \item action spaces,
    \item policies,
    \item trajectories,
    \item different formulations of return,
    \item the RL optimization problem,
    \item and value functions.
\end{itemize}

\begin{equation}
    R(\tau)=\sum_{t=0}^{T} r_t
\end{equation}

\begin{equation}
    R(\tau)=\sum_{t=0}^{\infty} \gamma^t r_t
\end{equation}
% $\displaystyle \mathop{\mathbb{E}}_{x\in A}$
\begin{equation}
    \gls{value}^{\gls{policy}}(\gls{state})
    =
        {\displaystyle \mathop{\mathbb{E}}_{\gls{trajectory}\sim\gls{policy}}}
    \lbrack R(\gls{trajectory})~|~\gls{state0}=\gls{state}\rbrack,
\end{equation}

\begin{equation}
    \gls{Q}^{\gls{policy}}(\gls{state}, \gls{action})
    =
        {\displaystyle \mathop{\mathbb{E}}_{\gls{trajectory}\sim\gls{policy}}}
    \lbrack R(\gls{trajectory})~|~\gls{state0}=\gls{state},~\gls{action0}=\gls{action}\rbrack,
\end{equation}

\begin{equation}
    \gls{value}^{*}(\gls{state})
    =
    \max_{\gls{policy}}{\displaystyle \mathop{\mathbb{E}}_{\gls{trajectory}\sim\gls{policy}}}
    \lbrack R(\gls{trajectory})~|~\gls{state0}=\gls{state}\rbrack,
\end{equation}

\subsection{Taxonomy of Reinforcement Learning}
%%%%%%%%%%%%%%%%%%%%%%%%%%%%%%%%%%%%%%%%%%%%%%%%%%%%%%%%%%%%%%%%%%%%%%%%%%%%%%%%

\subsection{Value-based methods}
%%%%%%%%%%%%%%%%%%%%%%%%%%%%%%%%%%%%%%%%%%%%%%%%%%%%%%%%%%%%%%%%%%%%%%%%%%%%%%%%

\subsection{Policy-based methods}
%%%%%%%%%%%%%%%%%%%%%%%%%%%%%%%%%%%%%%%%%%%%%%%%%%%%%%%%%%%%%%%%%%%%%%%%%%%%%%%%

\subsection{Policy gradient}
%%%%%%%%%%%%%%%%%%%%%%%%%%%%%%%%%%%%%%%%%%%%%%%%%%%%%%%%%%%%%%%%%%%%%%%%%%%%%%%%

\subsection{Deep deterministic policy gradient (DDPG)}
%%%%%%%%%%%%%%%%%%%%%%%%%%%%%%%%%%%%%%%%%%%%%%%%%%%%%%%%%%%%%%%%%%%%%%%%%%%%%%%%
