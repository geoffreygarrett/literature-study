\chapter{Introduction}

% Context of the research relates to the survival of humanity
Humanities capabilities to detect, track, and characterise \glspl{NEO} are of 
paramount importance for the preservation of all known life in the observable universe. \glspl{NEO} are a class of small Solar System bodies orbiting the Sun at roughly 0.98-1.3 times the \textit{Earth-Sun distance}, a distance used in the definition of an astronomical unit (AU). The term \textit{small body} refers to any celestial objects which avoided accretion by the Sun, a major planet, or their largest moons~\cite{Davidsson2021}. Our capabilities of \textit{detection} and \textit{characterisation} of \glspl{NEO} promise significant returns on investment in our progression towards their complete intelligence-based automation. % continuous progression/strives

% An event that immediately makes clear the importance of small body detection
The case of 2019 OK demonstrated a momentous failure in automated \textit{detection}, leaving scientists ``stunned [and in] true shock"~\cite{chiu_2019}.
``2019 OK", a small body approaching Earth, was discovered only \textbf{twenty-four hours} before its closest approach in 2019 when it was approximately 0.01 AU (around 1.5 million kilometres) away from Earth with an apparent magnitude of 14.7~\cite{IAU2019OK} (a measure of relative brightness visible with a 203-millimetre telescope aperture~\cite[p.~24]{North2014}). At its closest approach, it passed \comment{closest approach or flyby} Earth at a distance of around 70,000 kilometres, less than one-fifth of the Earth-Moon distance. Although the object's size was estimated to be between 57-130 metres in diameter~\cite{NASA2019}, the characterisation of its mass was not feasible due to the lack of observations of the object's gravitational interaction with other objects. However, given characteristics of other known small bodies, the effect of an Earth impactor comparable to ``2019 OK" could yield between 10-300 megatons of explosive energy~\cite{Cellino1999, Rumpf2017}. For reference, one megaton is equivalent to one million tons of TNT. The tsunami waves following the ``2004 Indian Ocean earthquake" (a.k.a. the ``Sumatra-Andaman earthquake") released around five megatons of energy resulting in the loss of more than 225,000 lives~\cite{Nirupama2006}. The ``Tsar Bomba", the most powerful nuclear weapon in history, had an explosive yield of 50 megatons~\cite{Khan2020}. This yield was around 1300 times that of ``Little Boy" and ``Fat Man", which were nuclear bombs fueled by highly enriched uranium and dropped on Hiroshima and Nagasaki, respectively, ending World War II~\cite{osti_1489669}. Given that an Earth impactor dimensionally similar to ``2019 OK" could yield between one-fourth and six times the explosive energy of the ``Tsar Bomba'', it is clear that a period of twenty-four hours between detection and closest approach is an unacceptable failure. There is subsequent consensus amongst scientists on the cause of this failure: objects approaching in some directions of the sky towards Earth can exhibit a slow apparent motion. In case studies similar to that of ``2019 OK", the apparent motion was as low as 0.1 degrees per day~\cite{Wainscoat2022}. 

Generally speaking, the ability to meet temporal requirements on the characterisation of a small body, in the case of a potential Earth impactor, is contingent on the timing of its detection. The earlier the detection, the greater the time budget to carry out characterisation, without which we cannot effectively modify an impactor's trajectory, or modify its structure such that it burns up in Earth's atmosphere. There have been many proposed ideas for deflecting an asteroid from an Earth-bound trajectory, however to date, none have been tested in space, and the feasibility of the proposed methods differs significantly~\cite{Harris2015}. There is however \glsposs{NASA} \gls{DART}, a mission that is currently on a 10-month journey to perform a kinetic impact manoeuvre into the secondary, Dimorphos, of the binary system, ``(65803) Didymos I Dimorphos". This exercise will be a test of \glsposs{NASA} capability and technologies in planetary defence, with mission success defined by a change in the binary orbital period of the system by at least 73 seconds, measured within an accuracy of 7.3 seconds or less. The mission follows a study performed by the John Hopkins Applied Physics Laboratory with support from members of \gls{NASA} in 2012~\cite{Cheng2012}. As mentioned earlier in the context of ``2019 OK", the characterisation of a small body's mass and structure can be better constrained through the observation of its interaction with other (small) bodies. This presents Dimorphos as a particularly attractive target for a deflection manoeuvre exercise, considering its proximity to Didymos, further supported by excellent Earth-based viewing conditions at the time of the planned impact (between 2022 September 25$^{\text{th}}$ and October 2$^{\text{nd}}$). The data returned by \gls{DART} will ultimately determine the momentum transfer efficiency of the kinetic impactor technique and characterize the effects of the impact~\cite{Cheng2012, Rivkin2021}.

There have been many missions to small celestial bodies pursued by humanity in our endeavour to better understand the formation of the Solar System and the conditions in the early solar nebula. The major categories of scientific questions, for which small body exploration bears the potential of answering, are ones addressing the conditions of the early solar nebular and planetesimal formation~\cite{Davidsson2021}. With limited resources for purely scientific missions, optimal characterization of small bodies is one of many obstacles to the pursuit of understanding the Solar System and the process of "planetesimal formation" that occurred about 4.5 billion years ago~\cite{Klahr2015}. With effective and optimal characterization techniques of small bodies, the feasibility of in-situ resource acquisition (e.g. ``asteroid mining") for construction, life support systems and propellants, also increases significantly. There is no contending that asteroid mining is a simple problem financially, it has significant cost implications, and there is no shortage of doubt as to whether this industry could be made profitable soon. Hein et al. focused their techno-economic analysis on the supply of water in space and the return of platinum to Earth and conclude that a key technical parameter for reaching break-even in a shorter time is the use of smaller, but increased numbers of spacecraft per mission ~\cite{Hein2020}. The proposed use of smaller satellites for \gls{NEO} missions is becoming increasingly popular in published research due to reduced costs and distributed risk of mission failure across multiple satellites as opposed to one~\cite{Wells2006, Laurin2008, Scott2013, Yu2014}.


The challenge is not to solve the problem


This work aims to elucidate and explore
existing literature defining the boundaries of, and intersections between, the
fields of space exploration and machine autonomy in the context of small body
characterization.




\cite{Rivkin2021}

% our survival: detection, tracking and characterization

% detection failure

% characterization dependent on detection

% characterization: survival, history of the solar system, estimation of economic resources for space missions and bases.



\cite{Marks2022}


% https://ssd.jpl.nasa.gov/tools/sbdb_lookup.html#/?sstr=3843336 database of small bodies from JPL.
% https://cneos.jpl.nasa.gov/nda/overview.html cool NASA app on NEO Deflection calculations.

%
%\begin{fancyquotes}
%    The sciences do not try to explain, they hardly even try to interpret, they
%    mainly make models. By a model is meant a mathematical construct which, with
%    the addition of certain verbal interpretations, describes observed
%    phenomena. The justification of such a mathematical construct is solely and
%    precisely that it is expected to work - that is correctly to describe
%    phenomena from a reasonably wide area. Furthermore, it must satisfy certain
%    aesthetic criteria - that is, in relation to how much it describes, it must
%    be rather simple. - John von Neumann
%\end{fancyquotes}


\section{Brief History of Spacecraft Missions to Small Celestial Bodies}

There have been many missions to small celestial bodies pursued by mankind in
our drive to better understand the formation of the Solar System and the
conditions in the early solar nebula. The term \textit{small body} refers to any
celestial objects which avoided accretion by the Sun, a major planet, or their
largest moons. The major categories of scientific questions which, for which
small body exploration bears promising of answering, are ones addressing
\textit{the conditions of the early solar nebular} and \textit{planetisimanl
    formation}~\cite{Davidsson2021}.

- NEAR Shoemaker on Eros
- NASA's Stardust mission: Annefrank, Wild 2 and Tempel 1
- Japan's Hayabusa Sampling Mission: Itokawa

Hayabusa (aka, MUSES-C) was a Japanese spacecraft designed to return samples
from the near-Earth asteroid Itokawa. It launched on May 9, 2003, and
successfully met up with Itokawa in September 2005. The spacecraft endured
multiple malfunctions during the mission but managed to finish most of its major
objectives. The spacecraft's samples returned to Earth on June 13, 2010, but it
took time for scientists to open its container and check for samples. Hayabusa
mission scientists confirmed in November 2010 that Hayabusa indeed picked up
samples of Itokawa.

Itokawa is a potentially hazardous asteroid that periodically crosses Earth's
orbit; that's one of the reasons this asteroid was chosen for close-up study.
It's about 1,150 feet (350 meters) in diameter and is classified as an S-type
asteroid. Images from the spacecraft showed few impact craters, although a
"rubble pile" appears on the surface.


- ESA's Rosetta Comet Mission

The European Space Agency's Rosetta spacecraft was a popular mission that
successfully made its way to a comet and landed a probe, called Philae, on the
object's surface. Rosetta launched on March 2, 2004, and made two asteroid
flybys before its last destination: Steins (September 2008) and Lutetia (July
2010).

When Rosetta reached Steins, the probe discovered that the object is a rare
E-type (enstatite) asteroid, meaning that it has iron-poor silicates on its
surface. The asteroid is roughly 4.1 miles (6.6 km) at its longest dimension and
is likely part of a larger object that broke apart. Rosetta spotted impact
craters on Steins' surface, and the space rock's measurements suggest that the
interior consists of rubble. The asteroid will likely disintegrate due to its
delicate interior.

- NASA's Dawn rises

NASA's Dawn mission launched on Sept. 27, 2007, to investigate two large members
of the asteroid belt: Ceres (a dwarf planet) and 4 Vesta (an asteroid). First,
the spacecraft came to Vesta, orbiting the asteroid between July 2011 and
December 2012. Dawn's next and final destination was Ceres, where it entered
orbit on March 6, 2015. (NASA also considered sending Dawn to visit a third
target but ultimately turned down the idea.) The mission ended in late 2018,
when the probe's hydrazine fuel will run out.

- Japan's lost Procyon

Japan's Procyon, also known as Proximate Object Close flyby with Optical
Navigation, launched with Hayabusa2 on Dec. 4, 2014. In 2016, the probe was
supposed to fly by asteroid 2000 DP107, but a problem with the ion-thruster
system on Procyon forced the Japan Aerospace Exploration Agency to abandon the
mission. Procyon did, however, catch a glimpse of Comet 67P, the destination of
the Rosetta mission.

- New Horizons explores Kuiper Belt object Arrokoth

NASA's New Horizons mission was designed to fly past Pluto, which it did in
2015. Because the spacecraft was in fine shape, mission personnel evaluated
other destinations that the probe had fuel to reach and decided to fly past an
object then known only as 2014 MU69, which had been discovered after the
spacecraft launched.

New Horizons' flyby determined that this Kuiper Belt object, now officially
called Arrokoth, was a contact binary, formed when two space rocks glide gently
into each other.


- NASA samples with OSIRIS-REx

NASA's OSIRIS-Rex (Origins, Spectral Interpretation, Resource Identification,
Security, Regolith Explorer) launched on Sept. 8, 2016, en route to Bennu, a
C-type asteroid. The spacecraft arrived at Bennu in August 2018 and spent nearly
two years studying the asteroid from orbit.

On Oct. 20, 2020, the OSIRIS-REx spacecraft conducted the key maneuver of its
mission, capturing a sample of the rocky world to bring back to Earth. The
spacecraft left Bennu in May 2021 and is scheduled to deliver its cargo in
September 2023.


%%%%%%%%%%%%%%%%%%%%%%%%%%%%%%%%%%%%%%%%%%%%%%%%%%%%%%%%%%%%%%%%%%%%%55  FUTURE

- NASA hits an asteroid with a DART
NASA's Double Asteroid Redirection Test (DART) will be a new kind of asteroid
mission for the agency. Instead of focusing on detailed science observations,
DART is a planetary defense mission dedicated to giving scientists their first
real-world data about how they might be able to deflect an asteroid headed for
Earth.

In late 2022, DART will arrive at an asteroid called Didymos, release a cubesat
to record the scene, then slam into Didymos' moon, Dimorphos. Scientists will
watch from Earth to see how much the impact tweaks the moon's orbit around the
larger asteroid.

Later this decade, a European Space Agency mission called Hera will head out to
Didymos as well to study the asteroid and crater after the dust has settled.


- Psyche, a metallic world to explore


In 2022, NASA will launch the Psyche mission to visit an asteroid also called
Psyche. The world is strangely metallic for an asteroid, leaving scientists with
a puzzle — and a hope that the body may turn out to be the bare core of a planet
that lost its rock. The spacecraft will launch in 2022 and reach its target
in 2026, then spend 21 months orbiting Psyche.