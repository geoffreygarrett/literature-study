\chapter{Introduction}

% Context of the research relates to the survival of humanity
Humanity's capabilities to detect, track, and characterise \glspl{NEO} are of 
paramount importance for the preservation of all known life in the observable universe. \glspl{NEO} are a class of small Solar System bodies orbiting the Sun at roughly 0.98--1.3 times the \textit{Earth-Sun distance}, a distance used in the definition of an astronomical unit (AU). The term \textit{small body} refers to any celestial objects which avoided accretion by the Sun, a major planet, or their largest moons~\cite{Davidsson2021}. Our capabilities of \textit{detection} and \textit{characterisation} of \glspl{NEO} promise significant returns on investment in our progression towards their complete intelligence-based automation. % continuous progression/strives

% An event that immediately makes clear the importance of small body detection
The case of 2019 OK demonstrated a momentous failure in automated \textit{detection}, leaving scientists ``stunned [and in] true shock''~\cite{chiu_2019}. ``2019 OK'', a small body approaching Earth, was discovered only \textbf{twenty-four hours} before its closest approach in 2019 when it was approximately 0.01 AU (around 1.5 million kilometres) away from Earth with an apparent magnitude of 14.7~\cite{IAU2019OK} (a measure of relative brightness visible with a 203-millimetre telescope aperture~\cite[p.~24]{North2014}). At its closest approach, it passed Earth at a distance of around 70,000 kilometres, less than one-fifth of the Earth-Moon distance. Although the object's size was estimated to be between 57--130 metres in diameter~\cite{NASA2019}, the characterisation of its mass was not feasible due to the lack of observations of the object's gravitational interaction with other objects. However, given characteristics of other known small bodies, the effect of an Earth impactor comparable to ``2019 OK'' could yield between 10--300 megatons of explosive energy~\cite{Cellino1999, Rumpf2017}. One megaton of explosive energy yield equals that of a million tons of \gls{TNT}. The tsunami waves following the ``2004 Indian Ocean earthquake'' (a.k.a. the ``Sumatra-Andaman earthquake'') released around five megatons of energy resulting in the loss of more than 225,000 lives~\cite{Nirupama2006}. The ``Tsar Bomba'', the most powerful nuclear weapon in history, had an explosive yield of 50 megatons~\cite{Khan2020}. This yield was around 1300 times that of ``Little Boy'' and ``Fat Man'', which were nuclear bombs fueled by highly enriched uranium and dropped on Hiroshima and Nagasaki, respectively, bringing an end to World War II~\cite{malik1985}. Considering that an Earth impactor dimensionally similar to ``2019 OK'' could yield between one-fourth and six times the explosive energy of the ``Tsar Bomba'', it is clear that a period of twenty-four hours between detection and closest approach is an unacceptable failure. There is subsequent consensus amongst scientists on the cause of this failure: objects approaching in some directions of the sky towards Earth can exhibit a slow apparent motion. In case studies representative of ``2019 OK'', the apparent motion was as low as 0.1 degrees per day~\cite{Wainscoat2022}. 

Generally speaking, the ability to meet temporal requirements on the characterisation of a small body, in the case of a potential Earth impactor, is contingent on the timing of its detection. The earlier the detection, the greater the time budget to carry out characterisation, without which we cannot confidently modify an impactor's trajectory or structure such that it avoids Earth or burns up in its atmosphere. There have been many proposed ideas for deflecting an asteroid from an Earth-bound trajectory; however, to date, none have been tested in space, and the feasibility across the set of proposed methods exhibits high variance~\cite{Harris2015}. There is, however, \glsposs{NASA} \gls{DART}, a mission that is currently on a 10-month journey to perform a kinetic impact manoeuvre into the secondary, Dimorphos, of the binary system, ``(65803) Didymos I Dimorphos''. This exercise will test \glsposs{NASA} capability and technologies in planetary defence, with mission success defined by a change in the binary orbital period of the system by at least 73 seconds, measured within an accuracy of 7.3 seconds or less. The mission follows a study performed by the John Hopkins Applied Physics Laboratory with support from members of \gls{NASA} in 2012~\cite{Cheng2012}. As mentioned earlier, in the context of ``2019 OK'', the characterisation of a small body's mass, more precisely its physical structure, can be better constrained by observing its interaction with other bodies. Therefore, Dimorphos is a well-suited target for a highly observable deflection manoeuvre exercise, considering its proximity to Didymos, further supported by favourable Earth-based viewing conditions at the time of the planned impact: between September 25\textsuperscript{th} and October 2\textsuperscript{nd} of 2022. The data returned by \gls{DART} will ultimately determine the momentum transfer efficiency of the kinetic impactor technique and characterise the overall effects of the impact~\cite{Cheng2012, Rivkin2021}. The momentum transfer efficiency is ubiquitously given the symbol $\beta$ in the field of asteroid evolution studies and is arguably the most significant factor in attempts to deflect an asteroid from an Earth-bound trajectory. An axiomatic example characteristic of an asteroid, resulting in the variability of $\beta$, is whether it is porous or rocky. It is easier to impart momentum onto a rocky asteroid than a porous one, the latter exhibiting wasted momentum in its deformation rather than its deflection~\cite{Holsapple2012}. Therefore, our efficacy in deflecting an Earth-bound asteroid depends on the degree to which we can characterise its physical structure prior to the manoeuvre.

There have been many missions to small celestial bodies pursued by humanity in our endeavour to understand better the formation of the Solar System and the conditions in the early solar nebula. The major categories of scientific questions, for which small body exploration bears the potential of answering, address the conditions of the early solar nebular and planetesimal formation~\cite{Davidsson2021}. With limited resources for purely scientific missions, optimal characterisation of small bodies is one of many obstacles to the pursuit of understanding the Solar System and the process of planetesimal formation that occurred about 4.5 billion years ago~\cite{Klahr2015}. With effective and optimal characterisation techniques of small bodies, the feasibility of in-situ resource acquisition (e.g. ``asteroid mining'') for construction, life support systems and propellants also increases significantly. There is no contending that asteroid mining is a complex problem financially, as it involves significant risk at considerable magnitudes of cost, and there is no shortage of doubt as to whether this industry could be profitable within a short time hereafter. Hein et al.~\cite{Hein2020} focused their techno-economic analysis on the water supply in space and platinum transported to Earth. They concluded that a key technical parameter for reaching break-even in a shorter time balances the use of smaller but increased numbers of spacecraft per mining mission. The proposed usage of smaller satellites for \gls{NEO} missions is becoming increasingly popular in published research due to reduced costs and distributed risk of mission failure across multiple satellites as opposed to one~\cite{Wells2006, Laurin2008, Scott2013, Yu2014}. 

This work aims to elucidate and explore existing literature defining the boundaries of, and intersections between, the fields of space exploration and machine autonomy in the context of small body characterisation. This work will focus on how advances in autonomy may enable or enhance the yield of future space missions to small bodies, identifying knowledge gaps in the current understanding of how to fully utilise advances in autonomy for space missions to small bodies. In addition, this work shall contribute to the foundations of knowledge required for developing an architecture that would enable a spacecraft swarm to characterise a small body. The concerned notions of optimality are related to minimising risk and maximised resolution of characteristics. The balance of the optimality between these two factors would be critical in the preparation phase of a kinetic impact manoeuvre for deflecting an Earth-bound object. Removed from the context of planetary defence, asteroid mining operations may primarily be driven by the mitigation of mission risk. In contrast, scientific missions in planetary science might opt to trade off slightly higher levels of mission risk if it means exceptional characterisation of a small body which may hold secrets of the early solar nebula or planetesimal formation.

The structure of this work is as follows: first, a brief overview of historic missions in space exploration which involved orbiting or interacting physically with small bodies in \autoref{sec:historic_small_body_missions}. This is followed by a review of the literature on machine learning fundamentals, deep learning and reinforcement learning techniques in \autoref{chap:machine_learning}. Specific attention will be 

%  and its applications to characterisation of small bodies. Next, a review of the literature on the use of machine learning to characterise small bodies. Finally, a review of the literature on the use of machine learning to characterise small bodies in the context of planetary defence.

% risk and its management in the context of space missions. Next, the concept of optimality is introduced and its application to autonomous space exploration is discussed. Finally, a case study is used to demonstrate how principles of optimality might be applied in the context of an Earth-deflection kinetic impactor mission.






\cite{Rivkin2021}

% our survival: detection, tracking and characterization

% detection failure

% characterization dependent on detection

% characterization: survival, history of the solar system, estimation of economic resources for space missions and bases.



\cite{Marks2022}


% https://ssd.jpl.nasa.gov/tools/sbdb_lookup.html#/?sstr=3843336 database of small bodies from JPL.
% https://cneos.jpl.nasa.gov/nda/overview.html cool NASA app on NEO Deflection calculations.

%
%\begin{fancyquotes}
%    The sciences do not try to explain, they hardly even try to interpret, they
%    mainly make models. By a model is meant a mathematical construct which, with
%    the addition of certain verbal interpretations, describes observed
%    phenomena. The justification of such a mathematical construct is solely and
%    precisely that it is expected to work - that is correctly to describe
%    phenomena from a reasonably wide area. Furthermore, it must satisfy certain
%    aesthetic criteria - that is, in relation to how much it describes, it must
%    be rather simple. - John von Neumann
%\end{fancyquotes}

